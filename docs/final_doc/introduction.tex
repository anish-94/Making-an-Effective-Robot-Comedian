
\section{Introduction to Project}
This project was requested by Dr. Heather Knight, a Computer Science professor at Oregon State University.
Dr. Knight purposed this project to be a pilot research project, to investigate how aspects of comedy can improve a robot's charisma in Human-Robot Interaction (HRI).


This project, if successful, can benefit and influence the fields of comedy and HRI by showing how certain features of a robot comedian can be conducive to creating an entertaining encounter between a human and a comedian or robot.

In stand-up comedy, comedians have to rely on their scripted jokes and their ability to improvise to have a successful performance. In many ways, a robot interaction can be compared to a stand-up set. Within the context of stand up comedy, a robot must be receptive to an audience and tell jokes to make an audience laugh.

Why is the field of robot comedy relevant and worth studying? With automation slowly replacing menial tasks in society, such as an ATM or an automated cashier, interactions with bots are going to be much more common. These machines are less engaging to interact with, but require reciprocity to obtain a goal. While these machines do improve ease of use and convenience, people are not as expressive towards the robot. Even if the machine is unsuccessful in completing its task, expressive robots are received in a more empathetic manner by the users {\cite{DesignExBeh:2017}}. The user’s perception of a robot’s expression is more significant than its true internal state when seeking to generate engaging interactions {\cite{KnightEightLessons:2011}}.

Dr. Heather Knight noted the importance of the audience recognizing the performer as a social being, rather than as a dull object. In this respect, artificial social intelligence can bring greater engagement to the realm of the theatre. A socially intelligent robot comedian can be crafted by utilizing non-verbal gestures and by conveying a sense of character through spontaneous interactions with the audience {\cite{KnightEightLessons:2011}}.  Additionally, a study by Katevas et al. {\cite{RobotComedyLab:2015}} evaluated the influence of non-verbal aspects of joke delivery. Knight also noted how physical presence and embodiment in a robot creates a more expressive and engaging interaction {\cite{KnightEightLessons:2011}}. In another study done by Sjöbergh and Araki {\cite{RobotsMakeThings:2008}}, having a robot tell a joke was found to generate more of an audience reaction than having the joke read by a human. However, this study evaluated joke performance by a robot, but not an entire stand-up set. To extend on these lines of research, we intend to focus on the effect of verbally and physically expressing robot character qualities during a stand-up performance.

\subsection{Team Members}
The members of our team are Kevin Talik, Anish Asrani, and Arthur Shing. Kevin acted as team leader, and focused on creating the adaptive portion of the robot comedian. Anish focused on creating the audience sensing and crowdwork portion of the robot comedian. Arthur focused on creating topical variations of robot and human versions of jokes, as well as animations for the robot comedian. Our client, Dr. Heather Knight, supervised and worked along with us in many parts of the project. She often provided guidance or help in areas of development that we had trouble with.  
