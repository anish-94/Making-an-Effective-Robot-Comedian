\documentclass[onecolumn, draftclsnofoot,10pt, compsoc]{IEEEtran}
\usepackage{graphicx}
\usepackage{url}
\usepackage{setspace}

\usepackage{geometry}
\geometry{textheight=9.5in, textwidth=7in}

% 1. Fill in these details
\def \CapstoneTeamName{		AKARobotics}
\def \CapstoneTeamNumber{		13}
\def \GroupMemberOne{			Kevin Talik}
\def \GroupMemberTwo{			Anish Asrani}
\def \GroupMemberThree{			Arthur Shing}
\def \CapstoneProjectName{		How to Build an Effective Robot Comedian}
\def \CapstoneSponsorCompany{	Oregon State University}
\def \CapstoneSponsorPerson{		Heather Knight}

% 2. Uncomment the appropriate line below so that the document type works
\def \DocType{		Problem Statement
				%Requirements Document
				%Technology Review
				%Design Document
				%Progress Report
				}

\newcommand{\NameSigPair}[1]{\par
\makebox[2.75in][r]{#1} \hfil 	\makebox[3.25in]{\makebox[2.25in]{\hrulefill} \hfill		\makebox[.75in]{\hrulefill}}
\par\vspace{-12pt} \textit{\tiny\noindent
\makebox[2.75in]{} \hfil		\makebox[3.25in]{\makebox[2.25in][r]{Signature} \hfill	\makebox[.75in][r]{Date}}}}
% 3. If the document is not to be signed, uncomment the RENEWcommand below
%\renewcommand{\NameSigPair}[1]{#1}

%%%%%%%%%%%%%%%%%%%%%%%%%%%%%%%%%%%%%%%
\begin{document}
\begin{titlepage}
    \pagenumbering{gobble}
    \begin{singlespace}
    	% \includegraphics[height=4cm]{coe_v_spot1}
        \hfill
        % 4. If you have a logo, use this includegraphics command to put it on the coversheet.
        %\includegraphics[height=4cm]{CompanyLogo}
        \par\vspace{.2in}
        \centering
        \scshape{
            \huge CS Capstone \DocType \par
            {\large\today}\par
            \vspace{.5in}
            \textbf{\Huge\CapstoneProjectName}\par
            \vfill
            {\large Prepared for}\par
            \Huge \CapstoneSponsorCompany\par
            \vspace{5pt}
            {\Large\NameSigPair{\CapstoneSponsorPerson}\par}
            {\large Prepared by }\par
            Group\CapstoneTeamNumber\par
            % 5. comment out the line below this one if you do not wish to name your team
            \CapstoneTeamName\par
            \vspace{5pt}
            {\Large
                \NameSigPair{\GroupMemberOne}\par
                \NameSigPair{\GroupMemberTwo}\par
                \NameSigPair{\GroupMemberThree}\par
            }
            \vspace{20pt}
        }
        \begin{abstract}
        % 6. Fill in your abstract
				This project is about creating an effective robot comedian.
				The following is a brief statement of the project's problem, proposed solution, and performance metrics.
				The problem described is about the lack of a sense of liveliness in robotic interaction.
				In solution, the authors aim to create a robot comedian that reacts to audience reaction.
				The effectiveness of this will be measured after a series of studies on a live audience.
        \end{abstract}
    \end{singlespace}
\end{titlepage}
\newpage
\pagenumbering{arabic}
\tableofcontents
% 7. uncomment this (if applicable). Consider adding a page break.
%\listoffigures
%\listoftables
\clearpage

% 8. now you write!
\section{Problem Statement}

\subsection{The Problem}

Robots and comedy are perhaps two subjects that could not seem to be further apart from each other.
On one hand, improv comedians thrive off of a freeform style that works and flows with their environment.
For instance, comedians on the show "Whose Line Is It Anyways?" use whatever creative juices they have to create jokes, and often refer to inside jokes that the audience would understand.
On the other hand, even the term "robot" itself implies a rigid, uncompromising mechanical behavior.
However, in most comedy, jokes and comic effects are scripted and pre-prepared.
For instance, in stand-up comedy, the comedian must practice and prepare their jokes, while being attentive to the audience, so that deliveries and jokes can be timed or changed for greater humorous effect.
Like stand-up comedy, flexibility in robot interaction could convey a greater sense of liveliness.

** Insert things about the problem, after we meet with Heather **

The goal in this project is to improve the flexibility of robotic interaction.

\subsection{The Proposed Solution}

% Talk about proposed solution here

To solve this problem of \_, we will aim to create a set lists of jokes, with parameters attached to each joke.
Possible parameters include audience age, other demographics, and types of humor.
The bot will be able to deviate from its set list depending on audience reactions to the previous jokes.
*** Insert other proposed solutions after meeting with Heather, based on what is possible ***

Live performances of the bot will be done to attain data on audience feedback and on joke quality.


% TODO: From description page
% Primary contributions will be robot behavior design and testing with people. Students are expecting to produce a parameterizable live performance with a robot that will be performed with a human audience (with varying parameters) several times.
% Objectives:
% -Crafting a flexible robot set list with audience sensing
% -Conveying Authenticity and Liveness
% -Integration of Robot Character
% -Audience Sensing

\subsection{Performance Metrics}

I will insert more here once we meet with Heather to discuss performance metrics.
For now, we will know the project is completed when the live audience tests are finished, and our findings are put on paper.
A NAO bot with a set list of jokes that can deviate from what was pre-prepared based on audience reaction should also be delivered.

% Performance metrics: Tell how you will know when you have completed the project.
% Metrics help you and your client agree on what successful completion
% (e.g., faster, cheaper, easier to use, "a working prototype," a complete white paper with research results)
% of the project looks like.

% TODO: From description page
% 1. Performance algorithms
% 2. Performance sequences and bits
% 3. Simple audience model
% 4. Series of studies with live audiences



\end{document}
