
\subsection{Kevin Talik}
	\subsubsection{Fall Term}
	\begin{itemize}
		\item{Week 1-2} \\
			Got project this week; Problem statement due October 10; email client and team . 
			We set up a slack channel ( akarobotics.slack.com ), and shared our availability. 
		\item{Week 3} \\
			We need to come up with a proposed solution that is specific enough for us to define what we intend to fix, but vague enough so that we can have some room for exploration (as this is a research project). 
			We need to make sure that our problem statement reflects that stand up comedy is a metaphor for human to computer interaction. 

		\item{Week 4} \\
		Make sure github is organized and easy to understand for Kirsten and Ben. 
		LaTeX Font, IEEE standard font for the first page. Sans-serif not serif 
		\item{Week 5}
			\begin{itemize}
				\item \textbf{Plans} \\
				Finish Scripts for local, remote and real world. 
				Explain research questions in scripts 
				Practice the nltk and choreograph 
				\item \textbf{Problems} \\
				Kirsten gave us a 92 on the problem statement, she is having difficulty understanding our research based goals. 
				\item \textbf{Progress} \\
				We are behind on the requirements, and need to focus on how this assignment will drive the research for the project.
				We need to focus on documentation quite a bit as well. Improve notes
				\item \textbf{Summary} \\
				Meet with Kirsten to help elaborate on research deliverables.  
				Organize research, look up ieee research. 
			\end{itemize}

		\item{Week 6}
			\begin{itemize}
				\item \textbf{Plans} \\
			Kirrsten is looking at our rough draft right now, and there will also be a rubric for the final that we can look at.
			Research Class this week. We talked about:What are you trying to solve? Can a robot dynamically interact to a crowd? Robot Character perceivable to audience? Adaptive content based from audience feedback?
				
				\item \textbf{Problems} \\
				Expectations are a medium concern. Research based work is something we dont have to worry about without an IRB.
				\item \textbf{Progress} \\
				
					Ben mentioned that our project is difficult, and is masters and PhD level work. Mcgrath and Kirsten are meeting with Heather to make sure that we have a reasonable amount of that can be completed during this school year. 
				\item \textbf{Summary} \\
					Our secondary objectives are going be structured around the dynamic nature of the crowd adaption. If we can come up with a clever solution to implement decision and dialogue choices, we can add it into our design of the machine, but we use the machine to perform the research 
				
			\end{itemize}

		\item{Week 7}
			\begin{itemize}
				\item \textbf{Plans} \\
				Tech review, look at technology for language processing.
				\item \textbf{Problems} \\
				ipsum
				\item \textbf{Progress} \\
				Markov Decision Process, MDP 

				https://www.cs.rice.edu/~vardi/dag01/givan1.pdf 

				Pykov, markov chains in Python  

				https://github.com/riccardoscalco/Pykov 

				A way to make state machine with probability transitions, based off of obvservations 
				\item \textbf{Summary} \\
				Further Designed ways for crowd to receive input.ould test with a couple people yelling at a robot, to test what high audio levels do to the input 
			\end{itemize}

		\item{Week 8}
   			\begin{itemize}
				\item \textbf{Plans} \\
				Use tech review as both tech and literature review. Make sure that my choice of tools is unbiased. None of the "i did it because i did it" sort of things.
				\item \textbf{Problems} \\
				Current view of the tech does not describe the technology well with the features.
				\item \textbf{Progress} \\
				Added more AI reviews of tech that I can use:
				Tensor Flow, pytorch, pykov, NLTK, CFG.
				\item \textbf{Summary} \\
				Looked at feedback methods of audio sensors, and AI tools. Language processing may fall out of scope. Its a lot of data.
				Use timelines to implement jokes, break jokes into one sentence in each box, put waits and ambient movement in each joke. Make two timeline boxes to vary the joke\_robot and joke\_human 
			\end{itemize}
		\item{Week 9}
			\begin{itemize}
				\item \textbf{Plans} \\
			testt Say and branching dialogue in choregraphe 
				\item \textbf{Problems} \\
					Organizing large choregraphe functions (timelines are going to be better for things that depend on time, obviously you dunce) 			
				\item \textbf{Progress} \\
			Got say boxes and branching to work, but in a messy way	
				\item \textbf{Summary} \\
				Movements and jokes are going to be best implemented in choregraphe. Algorithm may work by just taking text input, returning text input and the branches just follow the path
			\end{itemize}
		\item{Week 10}
				\begin{itemize}
				\item \textbf{Plans} \\
				Finish progress report
				\item \textbf{Problems} \\
				We should be worried about the progress report, just because it is a lot of work due in a short time.
				\item \textbf{Progress} \\
				Finished Progress report, wrote some comedic devices for comedy. Like, observational humor, Jobs jokes, like uber driver, dance lessons, unemployment
				\item \textbf{Summary} \\
				Further looked at types of jokes, finished Progress report
			\end{itemize}
	\end{itemize}
	\subsubsection{Winter Term}
	\begin{itemize}
		\item{Week 1}
			\begin{itemize}
				\item \textbf{Plans} \\
				Meet with heather, plan implementation term.
				\item \textbf{Problems} \\
					Branching will have to be done in choreographe, does not test well this way (in a physical manner)
					Hand drawn picture notes are hard to put into one note.
				\item \textbf{Progress} \\
				Heather called me the glue, Arthur the Lead voice, and Anish the perception lead.
				\item \textbf{Summary} \\
				For crowd work, look at Bill Burr getting heckled by a blind guy.

			\end{itemize}
		\item{Week 2}
			\begin{itemize}
				\item \textbf{Plans} \\
				Look at Bayes Classification
				\item \textbf{Problems} \\
				More specific so that we can work on improving parts. If I write jokes, they cant be for fun, they have to be specific to the research. 
				\item \textbf{Progress} \\
				Layed out different types of Adaptation. Correct adaptation, incorrect adaptation, and random adaptation.
				\item \textbf{Summary} \\
				Crowd report needs tons of qualities, yet it cant be a joke. Crowd needs to feel like robot is talking to them
				Categories of Jokes, categories of attributes, physicality, appropriateness of the audience.
			\end{itemize}
		\item{Week 3}
			\begin{itemize}
				\item \textbf{Plans} \\
				Meet with team in person to discuss how we will implement our comedian system
				\item \textbf{Problems} \\
				Our Sections are being completed, but they arent co-mingling because we made them alone, and didnt test together.
				\item \textbf{Progress} \\
				Formalized the Break a leg joke.
				Looked at pirate types of job jokes.
				Read about prosodic phrasing to time the jokes better.
				\item \textbf{Summary} \\
			Each of us are struggling with research goals and expectations, 

			and even though we can cover more ground by covering different topics, we 

			are not always on the same page with what each other are doing. 
			\end{itemize}
		\item{Week 4}
			\begin{itemize}
				\item \textbf{Plans} \\
				Writing more jokes for the robot.
				look at adaptation use cases
				\item \textbf{Problems} \\
				Romance jokes are funny, but not appropriate most of the time.
				"Things you can say about your router but not your girlfriend"

				\item \textbf{Progress} \\
				premise writing: Ginger has six fingers and cant work any job that requires hands. Maybe Ginger had a long term relationship with a slow cooker. Had a hot and steamy relationship with a dishwasher. Ginger being a DJ job. "I guess I am doing well at parking, because someone left a note on my car that said 'parking fine'"
				\item \textbf{Summary} \\
				Ginger takling is not as funny as ginger moving.
			\end{itemize}
		\item{Week 5}
			\begin{itemize}
				\item \textbf{Plans} \\
				Work on Bayes Net, implementing AL memory functionality of choreographe, finish haikubot.py
				\item \textbf{Problems} \\
				Heather left the country with the robot without telling us.
				\item \textbf{Progress} \\
					Made a random poem generator from books by HP lovecraft:

					Here are some examples:

		Still more I scraped, and then on some level beach. 

		It is still extant. 

		He had not memorized. 


		-----

		About the period of this material I cannot hope to understand. 

		Of genuine blood there was no whitish deposit whatever. 

		Night would soon fall, and it can't multiply. 

		-----

		Carrington Harris, last of the visible ritual. 

		I had witnessed things more potent than luminosity. 

		It was no one in Yog-Sothoth. 

		-----

		Now the irony is seldom absent. 

		He reeled, and would have let him live permanently with Peleg. 

		Man rules now where They shall break through again. 

				\item \textbf{Summary} \\
				Text generation is funny, need to look more at bayes 
			\end{itemize}
		\item{Week 6}
			\begin{itemize}
				\item \textbf{Plans} \\
				Look further into prosdy, Write more on midterm design.
				\item \textbf{Problems} \\
				The visible scope of choreographe blocks has a terrible implementation, and our behavior functions are unpredictable at this point.
				\item \textbf{Progress} \\
			Prosody, or intonation is the rhythm and emphasis of a sentence. 

			 

			You know. \textit{I} don’t. \ So don’t ask me.

			You know. I \textit{don’t}. \ As a matter of fact, I really don’t.

			You \textit{know} I don’t. \ You know that I don’t.
				\item \textbf{Summary} \\
				Joke Inventory: 2 carbon dating jokes, 2 dial up jokes, 1 break a leg joke, 2 last term random jokes, Autonomous car joke. We are launching jokes from one file, as to look like one performance.
			\end{itemize}
		\item{Week 7}
			\begin{itemize}
				\item \textbf{To do} \\
					Write Break a leg bit, get joke obbject to put into queue.
				\item \textbf{Progress} \\
					Finalized the break a leg joke, got a global queue for jokes initializing. Anish has head tracking implemented. Got a queue working for this.
				\item \textbf{Problems} \\
					Choreographe has a clunky mechanism for writing custom made objects, and python script boxes dont exit flow correctly
			\end{itemize}
		\item{Week 8}
			\begin{itemize}
				\item \textbf{Plans} \\
				Read more about the theory of chatbots. Think: Does showing a character make it more funny, because its showing true to itself?
				\item \textbf{Problems} \\
				Midterm Week, team is busy
				\item \textbf{Progress} \\
				Read this: https://apps.worldwritable.com/tutorials/chatbot/
				\item \textbf{Summary} \\
				Read about the first chatbot from Joseph Weizenbaum.
				I love this quote:
				“It is said that to explain is to explain away. This maxim is nowhere so well fulfilled as in the area of computer programming, especially in what is called heuristic programming and artificial intelligence…Once a particular program is unmasked, once its inner workings are explained in language sufficiently plain to induce understanding, its magic crumbles away; it stands revealed as a mere collection of procedures, each quite comprehensible. The observer says to himself, I could have written that.” 

				—Joseph Weizenbaum, ELIZA (1966) 

			\end{itemize}
		\item{Week 9}
			\begin{itemize}
				\item \textbf{Plans} \\
				Work with anish to get sound report
				\item \textbf{Problems} \\
				Anish could not meet to get sound report working
				\item \textbf{Progress} \\
				Not much progress, audience adaptation is spoofing in bayes net
				\item \textbf{Summary} \\
				we need to meet more frequently in person, 
			\end{itemize}
		\item{Week 10}
			\begin{itemize}
				\item \textbf{Plans} \\
				Identify matching and subscribing to events on naoqi API. Work on final report for winter term
				\item \textbf{Problems} \\
				naoqi API is not used very frequently, making documentation sparce and difficult to understand
				\item \textbf{Progress} \\
				Got final report done for winter.
				Figured out how to wait for behaviors (processes) to finish. Its a busy loop, it uses a lot of power on the robot.
				This will keep checking if the "behavior ended" function call is put into ALmemory 
			\end{itemize}
	\end{itemize}

	\subsubsection{Spring Term}
	\begin{itemize}
		\item{Week 1}
			\begin{itemize}
				\item \textbf{Plans} \\
					Start tying up loose ends for full comedy show. Start building controllers for audio sensing
				\item \textbf{Progress} \\
					Arthur and I met up in Graf to work out some of the loose ends on the robot
					Fully animiated pirate joke.+Tested pickle serializing in choregraphe(it works) 

					+this means that we can port the AI from my laptop to local in the robot 

					+started stripping perform.py into the choregraphe version, 'chorePerform.py' 

					+Discovered reboot from cmd line is x1000 faster 

					+Completed 2 modules for CITI training 

					+IRB approval will be easy if we can get 'exempt' status 
				\item \textbf{Problems} \\
					+Make performance work from pickled joke database 

					+Reformat joke-adding scripts 

					+add script to send badjokes.p to NAO 

					+Write Self Depreciation Set? 

					+Need 2-3 jokes? 

					+We need joke responses in choregraphe 

					+Do more CITI training 
				\item \textbf{Summary} \\
					Our jokes arent as funny as we thought when we wrote them haha. Told heather about crowd controllers, going to finish my rapid prototyping.
			\end{itemize}

		\item{Week 2}
			\begin{itemize}
				\item \textbf{Plans} \\
					Practice studying for expo presentation
				\item \textbf{Progress} \\
					I like this idea: Using machines as an end to means, not a means to an end. Watched "Do You Trust This Computer?"
					Its about AI in common life. Here are some thoughts:Deep learning, and living in a world where we cant go back after we forget. Why practice surgery if a robot can never fail? 

					 

					Putting AI systems in drones. If they can think and interact, then they are. If they make the connections to kill people and learn, then it will kill. 

					Like china and russian putting these systems in drones and fucking people up 

					The social robots who learn from childrens faces 					 

					Google deep mind beating video games:\begin{verbatim} https://deepmind.com/research/alphago/ \end{verbatim}
"It can win at any game…in less than a minute" Elon Musk 

					 
				\item \textbf{Problems} \\
					I want to be a stand up comedian, and what happens if AI takes the joy out of my job?
				\item \textbf{Summary} \\
					Deeply consider the means that the AI ends bring. Its lazy coding.
			\end{itemize}

		\item{Week 3}
			\begin{itemize}
				\item \textbf{Plans} \\
					Work on controllers for crowd sensing.
				\item \textbf{Progress} \\
This week, Anish and I went to Hweekend and finished the controllers. I am ditching the idea of putting LEDs in the controllers, because it would cost too much and be distracting for the audience. 
				\item \textbf{Problems} \\

					The scripts I wrote in our github were not on our requirements, so I am not going to treat them like they are being graded. 

					 

					However, for the user documentation, we need to correlate the project functionality to the literal code that is included in our project. 
				\item \textbf{Summary} \\
					Controllers are coming along, but my code is not included in our final performance so Im not going to fix them.
					Going to bring up an Arcade machine Idea to put final robot into machine.
			\end{itemize}

		\item{Week 4}
			\begin{itemize}
				\item \textbf{Plans} \\
					PROGRESS REPORT IN A WEEK, HAMMER 
				\item \textbf{Progress} \\
						DONE: 

						-TCP connection over the RPI and the NAO works. We can launch behaviors remotely now! 

						-I edited the outline of my paper to not include human research results, so I think it will be more ethically complete 

							-I got arthur too meet with our group before noon. 
				\item \textbf{Problems} \\
					idk much about electrical engineering, and I tend to burn myself with soldering irons
				\item \textbf{Summary} \\
					The controllers can work, but we dont know if a user can use them correctly.
			\end{itemize}

		\item{Week 5}
			\begin{itemize}
				\item \textbf{Plans} \\
					Go to maker fair and test comedian and controllers
				\item \textbf{Progress/summary} \\
 I finished my final design of the controller, and I want to make an image to put onto the RPIs, so that when they fail during expo, we can reflash the units quickly. We also did the maker fare this last weekend, and we got crazy feedback about how to make our controllers better. Decided to just run with a two button design. It will be easier to branch between two options over three. Kids stop listening when you give them a controller. 

We can use the virtual robot in choreographe for a table top example of the robot we used. this will be more useful than an arcade, but the arcade would look aesthetically pleasing while it is on our table. 
			\end{itemize}

		\item{Week 6}
			\begin{itemize}
				\item \textbf{Plans} \\
					Prepare advertising pictures for capstone… Chairbot with a poster on it? 
				\item \textbf{Progress} \\
					Our team has made posters and fliers for the show. I think it was a good decision to remove the controllers, because they were too distracting. 

				\item \textbf{Problems} \\
Sound analytics have not been tested. I had the controllers finish, but decided against implementing them into the final show. When you give people a controller, they stop listening. 

 

Additionally, people will press buttons differently, and different controllers would have different responses. It would be difficult to correlate these responses to the crowd report. 

 

This also pulls the attention off of the work that my team has done. Same story with arcade.
				\item \textbf{Summary} \\
					Controllers will not make the final show, Advertising is done. This is Robot comedian is a ventrilloquist. We wrote jokes to put into the machine. 
			\end{itemize}

		\item{Week 7}
			\begin{itemize}
				\item \textbf{Plans} \\
					EXPO
				\item \textbf{Progress} \\
					We did expo
				\item \textbf{Problems} \\
					None, it went well.
				\item \textbf{Summary} \\
					We are going to run shows every 30 minutes, no break for lunch 

					Additionally, we are going to try and film and give out surveys after every show. 


					Since my focus was adaptation, I moved to the back of the room to keep myself from seeing  

					If anish was using adaptation algorithms, or just clicking on jokes
			\end{itemize}

		\item{Week 8}
			\begin{itemize}
				\item \textbf{Plans} \\
					We took the week off so that we didn’t have to listen to the same jokes for seven days 
			\end{itemize}

		\item{Week 9}
			\begin{itemize}
				\item \textbf{Plans} \\
					We need to finalize some of the data, and go through the surveys. 

					Heather wants us to go through all of our videos, and try and rate the average effectiveness of  

					Each joke that was told 
				\item \textbf{Progress} \\
					See Summary
				\item \textbf{Problems} \\
					Heather was upset that we did not get audio data on the NAO for all of the shows.  

					We have complete video for shows 4,5,6,8,9 though
				\item \textbf{Summary} \\
					FOUR JOKES PER PERFORMANCE 

					    Not including the crowd report, crowd work, or responses during branching 


						 ADAPTATION NOTES  

						     Ive noticed from the surveys that a few people did not like that their information was being recorded during the show (the crowd report at the end) 

							      SEED 

									We asked the crowd which topic they liked 

			 For shows 4,5,6,8,9 people chose the last response. Do people yell the loudest during the last one because they didn’t understand how to interact with the robot? 

								MIDDLE BIT 

						We tried comparing the audio information collected during the show, and comparing it to the noise that was made when the robot told the joke. It was always higher when we asked the audience to cheer. 

								It seems that clapping is louder than laughter, so its hard to compare these responses. 

								Crowd Report 

							 We have no data for this because the crowd report did not vary during the show 

							The algorithm chose the final crowd report "You thought my jokes were… not funny"  

							SELF DEPRECIATION IS THE BEST HUMOR DEVICE THAT WAS THERE imo 
			\end{itemize}

		\item{Week 10}
			\begin{itemize}
				\item \textbf{Plans} \\
					Make video, keep in mind the documentation 

					We need to rewatch the videos and find the average effectiveness 

					Heather thinks that data is the most important thing from the expo event 
				\item \textbf{Progress} \\
					Arthur and I went through all of the surveys. 

					 

					At a first glance, before making the charts, we had about 145 valid surveys (filled completely) 

					 

					Lots of comments. Thankfully, only two people were offended by the autonomous car joke. 

					 

					People seemed to enjoy the animation the most, and we could work on the timing quite a bit
				\item \textbf{Problems} \\
					Data parsing by human hands is slow and tedious 
				\item \textbf{Summary} \\
					    Adaptation 

						 Accurate data collection isnt that funny 

						 We found that people enjoyed the final crowd report of "You thought my jokes were not that funny" 

						Maybe its not good if the crowd is in charge of the show 

						People often chose the last topic option during the interactoin part 
			\end{itemize}

	\end{itemize}

	\pagebreak
