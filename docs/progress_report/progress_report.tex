\documentclass[onecolumn, draftclsnofoot,10pt, compsoc]{IEEEtran}
\usepackage{graphicx}
\usepackage{url}
\usepackage{setspace}

\usepackage{geometry}
\geometry{textheight=9.5in, textwidth=7in}

% 1. Fill in these details
\def \CapstoneTeamName{		AKA Robotics}
\def \CapstoneTeamNumber{		13}
\def \GroupMemberOne{     Arthur Shing}
\def \GroupMemberTwo{			Kevin Talik}
\def \GroupMemberThree{   Anish Asrani}
\def \CapstoneProjectName{		How to Make an Effective Robot Comedian}
\def \CapstoneSponsorCompany{	Oregon State University}
\def \CapstoneSponsorPerson{		Heather Knight}

% 2. Uncomment the appropriate line below so that the document type works
\def \DocType{		%Problem Statement
				%Requirements Document
				%Technology Review
				%Design Document
				Progress Report
				}
			
\newcommand{\NameSigPair}[1]{\par
\makebox[2.75in][r]{#1} \hfil 	\makebox[3.25in]{\makebox[2.25in]{\hrulefill} \hfill		\makebox[.75in]{\hrulefill}}
\par\vspace{-12pt} \textit{\tiny\noindent
\makebox[2.75in]{} \hfil		\makebox[3.25in]{\makebox[2.25in][r]{Signature} \hfill	\makebox[.75in][r]{Date}}}}
% 3. If the document is not to be signed, uncomment the RENEWcommand below
\renewcommand{\NameSigPair}[1]{#1}

%%%%%%%%%%%%%%%%%%%%%%%%%%%%%%%%%%%%%%%
\begin{document}

\bstctlcite{IEEEexample:BSTcontrol}
\begin{titlepage}
    \pagenumbering{gobble}
    \begin{singlespace}
        \hfill 
        % 4. If you have a logo, use this includegraphics command to put it on the coversheet.
        %\includegraphics[height=4cm]{CompanyLogo}   
        \par\vspace{.2in}
        \centering
        \scshape{
            \huge CS Capstone \DocType \par
            {\large\today}\par
            \vspace{.5in}
            \textbf{\Huge\CapstoneProjectName}\par
            \vfill
            {\large Prepared for}\par
            \Huge \CapstoneSponsorCompany\par
            \vspace{5pt}
            {\Large\NameSigPair{\CapstoneSponsorPerson}\par}
            {\large Prepared by }\par
            Group\CapstoneTeamNumber\par
            % 5. comment out the line below this one if you do not wish to name your team
            \CapstoneTeamName\par 
            \vspace{5pt}
            {\Large
                \NameSigPair{\GroupMemberOne}\par
                \NameSigPair{\GroupMemberTwo}\par
                \NameSigPair{\GroupMemberThree}\par
            }
            \vspace{20pt}
        }
        \begin{abstract}
       	Hello abstract
        \end{abstract}     
    \end{singlespace}
\end{titlepage}
\newpage
\pagenumbering{arabic}
\tableofcontents
% 7. uncomment this (if applicable). Consider adding a page break.
%\listoffigures
%\listoftables
\clearpage

% 8. now you write!
\section{Introduction/Problem}
A lot of the machines that surround us aren't very engaging to interact with. They serve their purpose, people get what they need, and the interaction is over. People do not consider robots as entities. That is the gap we are trying to close by performing stand-up comedy with a robot. Stand-up comedy is a casual and entertaining way for people to get more exposure to robots and see that robots are not just objects, but they are much more than that. 

An effective robot comedian should be able to entertain the audience and generate laughs. We hypothesize that the effectiveness is dependent on three major aspects - crowd-work or ability to integrate the audience, portraying a coherent and convincing character, and the ability to adapt the performance based on audience feedback. We will base our performances and studies around these three areas. 

\section{Progress So Far}

\section{Issues and Solutions}
One of the issues we have had so far is coming up with material to write for the robot. Unsurprisingly, being funny is not easy and a lot goes into making a reasonable script that works well with the robot and appeals to other people. Overcoming this will involve gaining a lot more experience with the robot, and exposure to different kinds of comedy.

Animating the robot can be tricky as well. Each keyframe needs to be stored in a timeline block and the animation needs to iterate through these frames at a reasonable rate. Push the animations too fast and the robot will topple over and fall flat on its face. Have the animations too slow, and it goes out of sync from the joke it is trying to say. This will come down to practice and working more and more with the robot to get an idea of what "works" and what does not. 
(Can someone put a diagram for timeline here?)



\section{Retrospective}

\subsection{Positives}

\subsection{Changes Required}

\subsection{Actions}

\pagebreak


\bibliographystyle{IEEEtran}
\bibliography{refs}

\end{document}
