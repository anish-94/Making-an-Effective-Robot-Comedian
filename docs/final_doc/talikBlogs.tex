
\subsection{Kevin Talik}
	\subsubsection{Fall Term}
	\begin{itemize}
		\item{Week 1-2} \\
			Got project this week; Problem statement due October 10; email client and team . 
			We set up a slack channel ( akarobotics.slack.com ), and shared our availability. 
		\item{Week 3} \\
			We need to come up with a proposed solution that is specific enough for us to define what we intend to fix, but vague enough so that we can have some room for exploration (as this is a research project). 
			We need to make sure that our problem statement reflects that stand up comedy is a metaphor for human to computer interaction. 

		\item{Week 4} \\
		Make sure github is organized and easy to understand for Kirsten and Ben. 
		LaTeX Font, IEEE standard font for the first page. Sans-serif not serif 
		\item{Week 5}
			\begin{itemize}
				\item \textbf{Plans} \\
				Finish Scripts for local, remote and real world. 
				Explain research questions in scripts 
				Practice the nltk and choreograph 
				\item \textbf{Problems} \\
				Kirsten gave us a 92 on the problem statement, she is having difficulty understanding our research based goals. 
				\item \textbf{Progress} \\
				We are behind on the requirements, and need to focus on how this assignment will drive the research for the project.
				We need to focus on documentation quite a bit as well. Improve notes
				\item \textbf{Summary} \\
				Meet with Kirsten to help elaborate on research deliverables.  
				Organize research, look up ieee research. 
			\end{itemize}

		\item{Week 6}
			\begin{itemize}
				\item \textbf{Plans} \\
			Kirrsten is looking at our rough draft right now, and there will also be a rubric for the final that we can look at.
			Research Class this week. We talked about:What are you trying to solve? Can a robot dynamically interact to a crowd? Robot Character perceivable to audience? Adaptive content based from audience feedback?
				
				\item \textbf{Problems} \\
				Expectations are a medium concern. Research based work is something we dont have to worry about without an IRB.
				\item \textbf{Progress} \\
				
					Ben mentioned that our project is difficult, and is masters and PhD level work. Mcgrath and Kirsten are meeting with Heather to make sure that we have a reasonable amount of that can be completed during this school year. 
				\item \textbf{Summary} \\
					Our secondary objectives are going be structured around the dynamic nature of the crowd adaption. If we can come up with a clever solution to implement decision and dialogue choices, we can add it into our design of the machine, but we use the machine to perform the research 
				
			\end{itemize}

		\item{Week 7}
			\begin{itemize}
				\item \textbf{Plans} \\
				Tech review, look at technology for language processing.
				\item \textbf{Problems} \\
				ipsum
				\item \textbf{Progress} \\
				Markov Decision Process, MDP 

				https://www.cs.rice.edu/~vardi/dag01/givan1.pdf 

				Pykov, markov chains in Python  

				https://github.com/riccardoscalco/Pykov 

				A way to make state machine with probability transitions, based off of obvservations 
				\item \textbf{Summary} \\
				Further Designed ways for crowd to receive input.ould test with a couple people yelling at a robot, to test what high audio levels do to the input 
			\end{itemize}

		\item{Week 8}
   			\begin{itemize}
				\item \textbf{Plans} \\
				Use tech review as both tech and literature review. Make sure that my choice of tools is unbiased. None of the "i did it because i did it" sort of things.
				\item \textbf{Problems} \\
				Current view of the tech does not describe the technology well with the features.
				\item \textbf{Progress} \\
				Added more AI reviews of tech that I can use:
				Tensor Flow, pytorch, pykov, NLTK, CFG.
				\item \textbf{Summary} \\
				Looked at feedback methods of audio sensors, and AI tools. Language processing may fall out of scope. Its a lot of data.
				Use timelines to implement jokes, break jokes into one sentence in each box, put waits and ambient movement in each joke. Make two timeline boxes to vary the joke\_robot and joke\_human 
			\end{itemize}
		\item{Week 9}
			\begin{itemize}
				\item \textbf{Plans} \\
			testt Say and branching dialogue in choregraphe 
				\item \textbf{Problems} \\
					Organizing large choregraphe functions (timelines are going to be better for things that depend on time, obviously you dunce) 			
				\item \textbf{Progress} \\
			Got say boxes and branching to work, but in a messy way	
				\item \textbf{Summary} \\
				Movements and jokes are going to be best implemented in choregraphe. Algorithm may work by just taking text input, returning text input and the branches just follow the path
			\end{itemize}
		\item{Week 10}
				\begin{itemize}
				\item \textbf{Plans} \\
				Finish progress report
				\item \textbf{Problems} \\
				We should be worried about the progress report, just because it is a lot of work due in a short time.
				\item \textbf{Progress} \\
				Finished Progress report, wrote some comedic devices for comedy. Like, observational humor, Jobs jokes, like uber driver, dance lessons, unemployment
				\item \textbf{Summary} \\
				Further looked at types of jokes, finished Progress report
			\end{itemize}
	\end{itemize}
	\subsubsection{Winter Term}
	\begin{itemize}
		\item{Week 1}
			\begin{itemize}
				\item \textbf{Plans} \\
				Meet with heather, plan implementation term.
				\item \textbf{Problems} \\
					Branching will have to be done in choreographe, does not test well this way (in a physical manner)
					Hand drawn picture notes are hard to put into one note.
				\item \textbf{Progress} \\
				Heather called me the glue, Arthur the Lead voice, and Anish the perception lead.
				\item \textbf{Summary} \\
				For crowd work, look at Bill Burr getting heckled by a blind guy.

			\end{itemize}
		\item{Week 2}
			\begin{itemize}
				\item \textbf{Plans} \\
				Look at Bayes Classification
				\item \textbf{Problems} \\
				More specific so that we can work on improving parts. If I write jokes, they cant be for fun, they have to be specific to the research. 
				\item \textbf{Progress} \\
				Layed out different types of Adaptation. Correct adaptation, incorrect adaptation, and random adaptation.
				\item \textbf{Summary} \\
				Crowd report needs tons of qualities, yet it cant be a joke. Crowd needs to feel like robot is talking to them
				Categories of Jokes, categories of attributes, physicality, appropriateness of the audience.
			\end{itemize}
		\item{Week 3}
			\begin{itemize}
				\item \textbf{Plans} \\
				Meet with team in person to discuss how we will implement our comedian system
				\item \textbf{Problems} \\
				Our Sections are being completed, but they arent co-mingling because we made them alone, and didnt test together.
				\item \textbf{Progress} \\
				Formalized the Break a leg joke.
				Looked at pirate types of job jokes.
				Read about prosodic phrasing to time the jokes better.
				\item \textbf{Summary} \\
			Each of us are struggling with research goals and expectations, 

			and even though we can cover more ground by covering different topics, we 

			are not always on the same page with what each other are doing. 
			\end{itemize}
		\item{Week 4}
			\begin{itemize}
				\item \textbf{Plans} \\
				Writing more jokes for the robot.
				look at adaptation use cases
				\item \textbf{Problems} \\
				Romance jokes are funny, but not appropriate most of the time.
				"Things you can say about your router but not your girlfriend"

				\item \textbf{Progress} \\
				premise writing: Ginger has six fingers and cant work any job that requires hands. Maybe Ginger had a long term relationship with a slow cooker. Had a hot and steamy relationship with a dishwasher. Ginger being a DJ job. "I guess I am doing well at parking, because someone left a note on my car that said 'parking fine'"
				\item \textbf{Summary} \\
				Ginger takling is not as funny as ginger moving.
			\end{itemize}
		\item{Week 5}
			\begin{itemize}
				\item \textbf{Plans} \\
				Work on Bayes Net, implementing AL memory functionality of choreographe, finish haikubot.py
				\item \textbf{Problems} \\
				Heather left the country with the robot without telling us.
				\item \textbf{Progress} \\
					Made a random poem generator from books by HP lovecraft:

					Here are some examples:

		Still more I scraped, and then on some level beach. 

		It is still extant. 

		He had not memorized. 


		-----

		About the period of this material I cannot hope to understand. 

		Of genuine blood there was no whitish deposit whatever. 

		Night would soon fall, and it can't multiply. 

		-----

		Carrington Harris, last of the visible ritual. 

		I had witnessed things more potent than luminosity. 

		It was no one in Yog-Sothoth. 

		-----

		Now the irony is seldom absent. 

		He reeled, and would have let him live permanently with Peleg. 

		Man rules now where They shall break through again. 

				\item \textbf{Summary} \\
				Text generation is funny, need to look more at bayes 
			\end{itemize}
		\item{Week 6}
			\begin{itemize}
				\item \textbf{Plans} \\
				Look further into prosdy, Write more on midterm design.
				\item \textbf{Problems} \\
				The visible scope of choreographe blocks has a terrible implementation, and our behavior functions are unpredictable at this point.
				\item \textbf{Progress} \\
			Prosody, or intonation is the rhythm and emphasis of a sentence. 

			 

			You know. \textit{I} don’t. \ So don’t ask me.

			You know. I \textit{don’t}. \ As a matter of fact, I really don’t.

			You \textit{know} I don’t. \ You know that I don’t.
				\item \textbf{Summary} \\
				Joke Inventory: 2 carbon dating jokes, 2 dial up jokes, 1 break a leg joke, 2 last term random jokes, Autonomous car joke. We are launching jokes from one file, as to look like one performance.
			\end{itemize}
		\item{Week 7}
			\begin{itemize}
				\item \textbf{To do} \\
					Write Break a leg bit, get joke obbject to put into queue.
				\item \textbf{Progress} \\
					Finalized the break a leg joke, got a global queue for jokes initializing. Anish has head tracking implemented. Got a queue working for this.
				\item \textbf{Problems} \\
					Choreographe has a clunky mechanism for writing custom made objects, and python script boxes dont exit flow correctly
			\end{itemize}
		\item{Week 8}
			\begin{itemize}
				\item \textbf{Plans} \\
				Read more about the theory of chatbots. Think: Does showing a character make it more funny, because its showing true to itself?
				\item \textbf{Problems} \\
				Midterm Week, team is busy
				\item \textbf{Progress} \\
				Read this: https://apps.worldwritable.com/tutorials/chatbot/
				\item \textbf{Summary} \\
				Read about the first chatbot from Joseph Weizenbaum.
				I love this quote:
				“It is said that to explain is to explain away. This maxim is nowhere so well fulfilled as in the area of computer programming, especially in what is called heuristic programming and artificial intelligence…Once a particular program is unmasked, once its inner workings are explained in language sufficiently plain to induce understanding, its magic crumbles away; it stands revealed as a mere collection of procedures, each quite comprehensible. The observer says to himself, I could have written that.” 

				—Joseph Weizenbaum, ELIZA (1966) 

			\end{itemize}
		\item{Week 9}
			\begin{itemize}
				\item \textbf{Plans} \\
				Work with anish to get sound report
				\item \textbf{Problems} \\
				Anish could not meet to get sound report working
				\item \textbf{Progress} \\
				Not much progress, audience adaptation is spoofing in bayes net
				\item \textbf{Summary} \\
				we need to meet more frequently in person, 
			\end{itemize}
		\item{Week 10}
			\begin{itemize}
				\item \textbf{Plans} \\
				Identify matching and subscribing to events on naoqi API. Work on final report for winter term
				\item \textbf{Problems} \\
				naoqi API is not used very frequently, making documentation sparce and difficult to understand
				\item \textbf{Progress} \\
				Got final report done for winter.
				Figured out how to wait for behaviors (processes) to finish. Its a busy loop, it uses a lot of power on the robot.
				This will keep checking if the "behavior ended" function call is put into ALmemory 
			\end{itemize}
	\end{itemize}

	\subsubsection{Spring Term}
	\begin{itemize}
		\item{Week 1}
			\begin{itemize}
				\item \textbf{Plans} \\
				\item \textbf{Progress} \\
				\item \textbf{Problems} \\
				\item \textbf{Summary} \\
			\begin{itemize}

		\item{Week 2}
			\begin{itemize}
				\item \textbf{Plans} \\
				\item \textbf{Progress} \\
				\item \textbf{Problems} \\
				\item \textbf{Summary} \\
			\begin{itemize}

		\item{Week 3}
			\begin{itemize}
				\item \textbf{Plans} \\
				\item \textbf{Progress} \\
				\item \textbf{Problems} \\
				\item \textbf{Summary} \\
			\begin{itemize}

		\item{Week 4}
			\begin{itemize}
				\item \textbf{Plans} \\
				\item \textbf{Progress} \\
				\item \textbf{Problems} \\
				\item \textbf{Summary} \\
			\begin{itemize}

		\item{Week 5}
			\begin{itemize}
				\item \textbf{Plans} \\
				\item \textbf{Progress} \\
				\item \textbf{Problems} \\
				\item \textbf{Summary} \\
			\begin{itemize}

		\item{Week 6}
			\begin{itemize}
				\item \textbf{Plans} \\
				\item \textbf{Progress} \\
				\item \textbf{Problems} \\
				\item \textbf{Summary} \\
			\begin{itemize}

		\item{Week 7}
			\begin{itemize}
				\item \textbf{Plans} \\
				\item \textbf{Progress} \\
				\item \textbf{Problems} \\
				\item \textbf{Summary} \\
			\begin{itemize}

		\item{Week 8}
			\begin{itemize}
				\item \textbf{Plans} \\
				\item \textbf{Progress} \\
				\item \textbf{Problems} \\
				\item \textbf{Summary} \\
			\begin{itemize}

		\item{Week 9}
			\begin{itemize}
				\item \textbf{Plans} \\
				\item \textbf{Progress} \\
				\item \textbf{Problems} \\
				\item \textbf{Summary} \\
			\begin{itemize}

		\item{Week 10}
			\begin{itemize}
				\item \textbf{Plans} \\
				\item \textbf{Progress} \\
				\item \textbf{Problems} \\
				\item \textbf{Summary} \\
			\begin{itemize}

	\end{itemize}

	\pagebreak
