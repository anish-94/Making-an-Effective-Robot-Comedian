\documentclass[journal, letterpaper, draftclsnofoot, onecolumn, 10pt]{IEEEtran}

\usepackage{graphicx}
\usepackage{amssymb}
\usepackage{amsmath}
\usepackage{amsthm}

\usepackage{alltt}
\usepackage{float}
\usepackage{color}
\usepackage{url}
\usepackage{listings}

\usepackage{balance}
\usepackage[TABBOTCAP, tight]{subfigure}
\usepackage{enumitem}
\usepackage{pstricks, pst-node}
\usepackage{placeins}
\usepackage{geometry}
\usepackage{hyperref}
\geometry{textheight=8.5in, textwidth=6in}

\lstset{
  language=C,                % choose the language of the code
  numbers=left,                   % where to put the line-numbers
  stepnumber=1,                   % the step between two line-numbers.
  numbersep=5pt,                  % how far the line-numbers are from the code
  backgroundcolor=\color{white},  % choose the background color. You must add \usepackage{color}
  showspaces=false,               % show spaces adding particular underscores
  showstringspaces=false,         % underline spaces within strings
  showtabs=false,                 % show tabs within strings adding particular underscores
  tabsize=8,                      % sets default tabsize to 2 spaces
  captionpos=b,                   % sets the caption-position to bottom
  breaklines=true,                % sets automatic line breaking
  breakatwhitespace=true,         % sets if automatic breaks should only happen at whitespace
  title=\lstname,                 % show the filename of files included with \lstinputlisting;
}

%random comment

\newcommand{\cred}[1]{{\color{red}#1}}
\newcommand{\cblue}[1]{{\color{blue}#1}}

\newcommand{\toc}{\tableofcontents}


\def\name{Kevin Talikk, Anish Asrani, and Arthur Shing}

%% The following metadata will show up in the PDF properties
\hypersetup{
   colorlinks = true,
   urlcolor = black,
   linkcolor = black,
   pdfauthor = {\name},
   pdfkeywords = {cs444 ``operating systems'' slob slab best-fit},
   pdftitle = {CS 444 Project 4: The SLOB SLAB},
   pdfsubject = {CS 444 Project 4},
   pdfpagemode = UseNone
}

\parindent = 0.0 in
\parskip = 0.1 in


\begin{document}
\title{How to make an effective robot comedian}
\author{Kevin Talikk, Anish Asrani, and Arthur Shing}
%TODO Insert class, title, term

\begin{titlepage}
    \pagenumbering{gobble}
    \centering
    \maketitle
    CS 461 - Senior Capstone\par
    Fall term\par
    \vspace{1cm}
    \begin{abstract}
      This is the abstract, 100-150 words long that summarizes the rest of the document
    \end{abstract}


\end{titlepage}
\pagenumbering{arabic}
% \tableofcontents
% \clearpage

\section{Problem Statement}

\subsection{The Problem}

% Define and describe the problem for a general but educated audience


\subsection{Proposed Solution}

% Talk about proposed solution here

\subsection{Performance Metrics}

% Performance metrics: Tell how you will know when you have completed the project.
% Metrics help you and your client agree on what successful completion
% (e.g., faster, cheaper, easier to use, "a working prototype," a complete white paper with research results)
% of the project looks like.


\FloatBarrier
\end{document}
