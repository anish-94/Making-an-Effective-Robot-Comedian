

\section{Introduction - Tech Review Anish}
During stand-up comedy, a performer's ability to influence the audience is vital to an entertaining performance. This involves correctly tying together various social signals such as body orientation, gesture, and gaze by both the performers and audience \cite{RobotComedyLab:2015}. Dr. Knight emphasized on the importance of such non-verbal interactions in order to deliver a successful performance \cite{KnightEightLessons:2011}. We want to add to this and incorporate crowd-work and audience interactions throughout the performances in order to help make the performance enjoyable and engaging for the audience.

\section{Individual Role in the Project}
One of the most important aspects of an effective performance is making the audience feel like they are part of the performance at all times. My role is to integrate this crowd-work throughout the performances. It can involve pointing at and talking to the audience which can be scripted, mostly during the intro and outro of the performance. It can be taken further by using sensors that will be triggered by certain audience actions and reactions.

\section{Literature Review}
The timing, frequency, and duration of a gesture relays a lot of information to the spectator. Dr. Knight conducted various performances with a robot, ranging from performing on stage with an audience to pre-mediated collisions with human environments, like "street performances." Over the course of these performances, there were eight major takeaways that will help make a robot performance entertaining. They were:

\subsection{Convey Intentionality}
Using relatable and appropriate gestures help the communication between the robot and the audience. Since these actions can be predicted by the audience, they consider the robot as an entity similar to them. Displaying empathy while performing also boosts the robot's relatability to the audience.

\subsection{No Mind Without Body}
Human expressions are derived from our physicality. Robots can also be capable of leveraging their embodiment to communicate on human terms. It was found that presence of a physical, embodied robot enabled more interaction as well as enjoyment of said interaction for humans. A robot not fully leveraging its physicality ends up losing a significant mode of communication and is also less expressive.

\subsection{Physicality and Motion}
The audience should be able to connect the robot's non-verbal behaviors to the words. The human brain maps the actions it sees on to itself and imagines itself doing it. This helps the audience put themselves in the shoes of the performer.

\subsection{Outward Emotional Communication Trumps Inward Experience}
The inner experience of the performer is trumped by the success of the outward intentionality conveyed. Most robots are designed to enhance, enable, or empower humans. Simplicity and clean physical design is often the clearest way to streamline communication of robot intention.

\subsection{Gulf between Props and Character}
Robots should be considered to be more like agents (entities) and less like props just standing up on stage. Until now, robots have been lacking in that department and there is a significant gap that exists. Robots lack believable and human-like actions. The various aspects of non-verbal communication like gestures have a significant impact on the robot not being considered an object.

\subsection{Good Actors Outweigh Bad Actors}
Multi-robot or human-robot teams have potential to deliver entertaining performances. Human actors can affect the audience's perception of the robot. They can also make up for the robot's unpredictability and lack of control.

\subsection{Acknowledging/Learning}
Human audiences are cognizant of human social behaviors. The audience can provide real time feedback. This feedback can be used to maximize the audience's enjoyment levels. The robot can constantly read these enjoyment levels and update the attributes of audience likes and dislikes. When delivering a joke, the audience should be given enough time to comprehend and process the joke. Starting the next joke early can break the flow and does not give the audience a chance to appreciate the joke and its delivery. The pause could be filled with the robot gazing around at the audience and posing. This helps develop a good rhythm for each joke.

\subsection{Humor Makes People Like the Robot }
Humor is one of the common grounds across all humans. When a robot performs comedy, and is able to match their sense of humor, it helps establish that common ground. If humor can help robot seem like ?one of us', that could be a significant leap to overcome the idea that robots are only props \cite{KnightEightLessons:2011}.


Katevas conducted studies in a similar fashion. His study hypothesized that interactional dynamics should be just as important to the mass interaction involved in performing comedy in front of a live audience. The interactional dynamics involved addressing the audience - including appropriately timed smiles or relevant gestures. The interactional procedures during a performance helps set the tone for the performance.

Robots provide a unique opportunity to experiment with the interactional processes. They can have a consistent routine while modifying the various aspects of delivery - body orientation, gaze, and gesture.

Embodied robots are likely the best way for the robot the catch the audience attention and make the audience pay attention to the robot. The robot used by Katevas et. al was a humanoid robot consisting of a robotic head, two arms with hands, the torso as well as two legs. The head had two rectangular LCD screens for eyes as well as LEDs on cheeks for expression \cite{RobotComedyLab:2015}.

\section{Methods}

Katevas' studies involved two performances. Each performance involved a compere doing the introductions and warming up the crowd for 10 minutes, followed by a human comedian performing for 13 minutes. Right after that, the robot comedian performed for 8 minutes. This format was used to widen the appeal of the event and to set up a stand-up comedy context. Each performance consisted of approximately 50 people in the audience. The sensors used to capture audience reactions and responses got data for approximately 20 people each performance. These sensors looked for various aspects about the audience including gender and age estimation. It also captured facial expressions and categorizing them into percentages of "happy", "sad", "angry", and "surprised". These facial expressions were used to update the audience model.

Punchlines were distinguished using a faster delivery followed by a short pause. The punchline was also followed by a gaze and a smile and sometimes laughter. The duration of the pause was determined by the feedback received from the joke - a longer pause if the audience is still laughing. While using gestures, a gesture pointing to the audience at certain times seemed to get the most positive feedback. The studies found that the enjoyment levels of the audience during the robot performance lied between the compere and the human comedian \cite{RobotComedyLab:2015}.

We could use the dynamic use of sensors to take into account and acknowledge when a delivery was successful or not. We will capture an initial audience model by performing a few "trial" jokes to get a grip on what kind of humor or topics the audience would enjoy. The robot's model for the audience could be acknowledged later to give the audience a feeling that they are being heard. While the sensors in our robot are not sophisticated enough to capture emotions from a crown of people, we will use audio feedback from the audience to update the audience model on the fly.

Dr. Knight's research was more varied. Her performances ranged from stage and audience performances to guerrilla theater performances (street performances). The research looked to add value to developing everyday robots in addition to the entertainment value. It is easy to survey people and learn more about human-robot interaction when there is a significant amount of people present in the audience.

Using the theater context also helps the development of social robots. As noted before, non-verbal expression plays a key role in understanding sociability. A robot's movement and engagement pattern impacts the people's interpretation of the robot's intention, capability, and state. Physical theater provides pre-processed methodologies for interpreting and communicating human non-verbal behaviors that can be portrayed on robots \cite{KnightEightLessons:2011}.

Capturing an audience model is not as feasible when performing guerrilla theater. The audience model would not be very accurate since the crowd is constantly changing. In such situations, the crowd-work would depend on gestures and being able to identify what jokes were successful.

\section{Conclusion}

Humor is one of the major emotions that is common ground for people from all backgrounds. Robots that have a grasp of humor would help bridge the gap present in human-robot interaction and bring people closer to robots. Stand-up performances provide a great platform for the robot to express its humor and show the audience what it is capable of. It will be vital to make the audience feel like a part of the performance, just like most successful stand-up comedians. In order to do this successfully, we need to take a lot from the research done in the past and build upon it.
