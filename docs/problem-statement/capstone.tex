\documentclass[onecolumn, draftclsnofoot,10pt, compsoc]{IEEEtran}
\usepackage{graphicx}
\usepackage{url}
\usepackage{setspace}

\usepackage{geometry}
\geometry{textheight=9.5in, textwidth=7in,margin=0.75in}

% 1. Fill in these details
\def \CapstoneTeamName{		AKARobotics}
\def \CapstoneTeamNumber{		13}
\def \GroupMemberOne{			Anish Asrani}
\def \GroupMemberTwo{			Kevin Talik}
\def \GroupMemberThree{			Arthur Shing}
\def \CapstoneProjectName{		How to Make an Effective Robot Comedian}
\def \CapstoneSponsorCompany{	Oregon State University}
\def \CapstoneSponsorPerson{		Heather Knight}

% 2. Uncomment the appropriate line below so that the document type works
\def \DocType{		Problem Statement
				%Requirements Document
				%Technology Review
				%Design Document
				%Progress Report
				}
			
\newcommand{\NameSigPair}[1]{\par
\makebox[2.75in][r]{#1} \hfil 	\makebox[3.25in]{\makebox[2.25in]{\hrulefill} \hfill		\makebox[.75in]{\hrulefill}}
\par\vspace{-12pt} \textit{\tiny\noindent
\makebox[2.75in]{} \hfil		\makebox[3.25in]{\makebox[2.25in][r]{Signature} \hfill	\makebox[.75in][r]{Date}}}}
% 3. If the document is not to be signed, uncomment the RENEWcommand below
%\renewcommand{\NameSigPair}[1]{#1}

%%%%%%%%%%%%%%%%%%%%%%%%%%%%%%%%%%%%%%%
\begin{document}
\begin{titlepage}
    \pagenumbering{gobble}
    \begin{singlespace}
        \hfill 
        % 4. If you have a logo, use this includegraphics command to put it on the coversheet.
        %\includegraphics[height=4cm]{CompanyLogo}   
        \par\vspace{.2in}
        \centering
        \scshape{
            \huge CS Capstone \DocType \par
            {\large\today}\par
            \vspace{.5in}
            \textbf{\Huge\CapstoneProjectName}\par
            \vfill
            {\large Prepared for}\par
            \Huge \CapstoneSponsorCompany\par
            \vspace{5pt}
            {\Large\NameSigPair{\CapstoneSponsorPerson}\par}
            {\large Prepared by }\par
            Group\CapstoneTeamNumber\par
            % 5. comment out the line below this one if you do not wish to name your team
            \CapstoneTeamName\par 
            \vspace{5pt}
            {\Large
                \NameSigPair{\GroupMemberOne}\par
                \NameSigPair{\GroupMemberTwo}\par
                \NameSigPair{\GroupMemberThree}\par
            }
            \vspace{20pt}
        }
        \begin{abstract}
        % 6. Fill in your abstract
        
        
        
        
        The purpose of this project is to make a robot that performs comedy. Robot interaction lacks the traits of human interaction. Character, authenticity, and liveliness are some traits of human interaction that enables people to stay invested in an interaction, and are important aspects of successful stand-up comedy. In an effort to create an effective robot comedian, these traits should be present in the robot. The challenge will be to make the robot feel and perform more like a human would while keeping the audience engaged. This can be overcome by considering audience feedback after every joke or story that is told and using artificial intelligence to adjust future jokes. We propose that the robot should be able to construct its own jokes according to these traits, and also in accordance with audience reaction. The robot’s performance will be measured using audience sensing tools.
        	
        \end{abstract}     
    \end{singlespace}
\end{titlepage}
\newpage
\pagenumbering{arabic}
\tableofcontents
% 7. uncomment this (if applicable). Consider adding a page break.
%\listoffigures
%\listoftables
\clearpage

\section{The Problem}



%[Problem questions]

%We want to use the metaphor of stand-up comedy to design a set list that can convey a sense of character to an audience. The character of the robot needs to show a sense of improvisation with dialogue and originality of content.


The key to keeping a human invested in a conversation is keeping them engaged and curious. An engaging conversation has content valuable to both people in the conversation, and upholding curiosity necessitates unknown content; people will be more interested in something they have not seen before. For a robot interaction, it is difficult to determine if what the robot is responding with is what the developer wrote, or what the machine created. 

The goal of research in AI and HCI is to create a sense of liveliness and flexibility in an interaction where there is none. 
A verbal interaction between a robot and a human can be compared to a stand-up comedy set. In comedy, comedians have to rely on their scripted jokes and their ability to improvise, the latter giving a sense of authenticity through their spontaneous interactions with the audience. With it’s interaction, the robot needs to convey a sense of character to assist in anthropomorphizing itself to portray authenticity . Comedians must also have an acute perception of the audience, and the flexibility to adjust their prepared performance for the audience. 

{\it{Character}} is a set of traits that describe and distinguish an individual. For a robot, the quality of a character needs to be discernible as individual and unique. It’s character traits should also simultaneously be capable of describing a {\it{human}} character.
A relatable character can connect to an audience and give a sense of familiarity, and the individuality of the robot's character can abstract the observer from the technology that is driving the machine’s decisions. 

Despite the ability of a well-crafted character to engage an audience from the mechanical wirings and programming of the robot, robot stand-up is ontologically different from a human doing stand-up. As such, there are certain things that being a {\it{robot on stage}} enables a robot comedian to have that human comedians do not. 

Additionally, a robot must “act” alive, responding to an audience in a way that is both comedic and convincingly natural, rather than mechanically moving on from one joke to another.




%We are trying to make a robot that performs stand-up comedy and entertain people. This should be done while making the robot as close to human as possible. The comedy routine will be scripted but there should be some sort of interactions between jokes to make the performance feel more real. The audience should feel engaged throughout the performance.



\section{Proposed Solution}

%Bot crafts its own jokes
%H1: Audience Engagement + Cooperation will improve quality of robot comedy 
We hypothesize that what makes an interaction feel authentic and engaging is the perceived character through improvisational dialogue. To implement this solution, we intend to make an Artificial Intelligence system that can analyze human provided topics, construct jokes, and deliver content from character decisions. The system will also be able to determine which portion of an interaction did well, and base future decisions from successful dialogue history.

The four core components of this system are the analysis of human topics (or intended subject), the content and joke creation algorithm, the delivery decision for the created content, and the robot's ability to learn from successful dialogue.

For initializing a dialogue from the robot, the system will need to describe and understand the context of an interaction. For example, if the person speaking to the robot mentions a subject, the robot needs to have some definitions of what a subject is,{ and the perceived character from the human}. The input can be something non-specific, such as laughter, or applause. If there is no specific noun, or topic, the robot should be able to have some experiences and stories written that it can talk about.

The joke creation algorithm will create a vocabulary and the conversation topic. For example, if the audience mentions an apple, the robot can reference an apple as round, fruit, red, seeds, etc. The model for humor, or what constitutes a joke as “funny”, will be quality of how an expectation is broken for a predicted outcome of a conversation. 

The delivery of a joke can make a joke well received (it is interpreted as “funny” for the audience), or it can illicit a negative response. The proposed solution for this system is dependent on a robot being able to exude a sense of character that is descriptively round; we want a human to be able to describe our robot with traits that project personality and charisma in a conversation.

After a joke has been told to an audience, the robot needs to be able to quantify and categorize how the joke was received to an audience, and if the response was net positive, or net negative. This metric (described more in performance metric), will further aid the robots character decision making, and how it will choose and deliver jokes. This will contribute to making the robot interact with a human audience more authentically.
 

\section{Performance Metrics}

% Did the bot accurately perceive audience reaction?
% Did the bot accurately respond to the perceived reaction?
% Response sensing: Cards, auditory sensing, maybe clickers
% Did we find if audience engagement/cooperation improves bot comedy quality?

% Did we have x amount of audience tests?

The robot should be able to sense the audience by using audio cues and analyzing laughter (e.g. read cue cards that will be provided to the audience). If the audience is enjoying themselves, the robot was successful. If the robot receives a negative response and the audience isn’t enjoying themselves, it should change its content accordingly to impress the audience. The goal is to keep the audience captivated. This partially involves writing a good variety of jokes and giving it to the robot, and the robot improvising to deliver the most appropriate jokes at every point. How the robot changes its tone and genre of jokes should be tested as well. 

There should be ways to test the robot even when there is no audience available. This could be done by feeding the response inputs manually and testing how it changed its jokes for different kinds of responses. 

Every story or joke that it is telling should have some structure to it. It should set-up the joke and allow the audience to visualize what is going on, describe it, then deliver the punchline. The jokes it makes should make use of the fact that it is a robot. A robot would potentially face different problems than what humans do. The audience would enjoy hearing things from a perspective they haven’t seen before. While doing this, it should also feel human. The robot should not feel flat and monotonous (e.g. Siri). The challenge will be to find the right balance between both of them.

The audience could be surveyed after every performance and ask them to describe the robot. These surveys could then be used to improve the robot little by little after every performance. 

\end{document}