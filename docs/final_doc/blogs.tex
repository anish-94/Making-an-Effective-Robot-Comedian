
\section{Weekly Blog Posts}
	\subsection{Kevin Talik}
	\subsubsection{Fall Term}
	\begin{itemize}
		\item{Week 1-2} \\
			Got project this week; Problem statement due October 10; email client and team . 
			We set up a slack channel ( akarobotics.slack.com ), and shared our availability. 
		\item{Week 3} \\
			We need to come up with a proposed solution that is specific enough for us to define what we intend to fix, but vague enough so that we can have some room for exploration (as this is a research project). 
			We need to make sure that our problem statement reflects that stand up comedy is a metaphor for human to computer interaction. 

		\item{Week 4} \\
		Make sure github is organized and easy to understand for Kirsten and Ben. 
		LaTeX Font, IEEE standard font for the first page. Sans-serif not serif 
		\item{Week 5}
			\begin{itemize}
				\item \textbf{Plans} \\
				Finish Scripts for local, remote and real world. 
				Explain research questions in scripts 
				Practice the nltk and choreograph 
				\item \textbf{Problems} \\
				Kirsten gave us a 92 on the problem statement, she is having difficulty understanding our research based goals. 
				\item \textbf{Progress} \\
				We are behind on the requirements, and need to focus on how this assignment will drive the research for the project.
				We need to focus on documentation quite a bit as well. Improve notes
				\item \textbf{Summary} \\
				Meet with Kirsten to help elaborate on research deliverables.  
				Organize research, look up ieee research. 
			\end{itemize}

		\item{Week 6}
			\begin{itemize}
				\item \textbf{Plans} \\
			Kirrsten is looking at our rough draft right now, and there will also be a rubric for the final that we can look at.
			Research Class this week. We talked about:What are you trying to solve? Can a robot dynamically interact to a crowd? Robot Character perceivable to audience? Adaptive content based from audience feedback?
				
				\item \textbf{Problems} \\
				Expectations are a medium concern. Research based work is something we dont have to worry about without an IRB.
				\item \textbf{Progress} \\
				
					Ben mentioned that our project is difficult, and is masters and PhD level work. Mcgrath and Kirsten are meeting with Heather to make sure that we have a reasonable amount of that can be completed during this school year. 
				\item \textbf{Summary} \\
					Our secondary objectives are going be structured around the dynamic nature of the crowd adaption. If we can come up with a clever solution to implement decision and dialogue choices, we can add it into our design of the machine, but we use the machine to perform the research 
				
			\end{itemize}

		\item{Week 7}
			\begin{itemize}
				\item \textbf{Plans} \\
				Tech review, look at technology for language processing.
				\item \textbf{Problems} \\
				ipsum
				\item \textbf{Progress} \\
				Markov Decision Process, MDP 

				https://www.cs.rice.edu/~vardi/dag01/givan1.pdf 

				Pykov, markov chains in Python  

				https://github.com/riccardoscalco/Pykov 

				A way to make state machine with probability transitions, based off of obvservations 
				\item \textbf{Summary} \\
				Further Designed ways for crowd to receive input.ould test with a couple people yelling at a robot, to test what high audio levels do to the input 
			\end{itemize}

		\item{Week 8}
   			\begin{itemize}
				\item \textbf{Plans} \\
				Use tech review as both tech and literature review. Make sure that my choice of tools is unbiased. None of the "i did it because i did it" sort of things.
				\item \textbf{Problems} \\
				Current view of the tech does not describe the technology well with the features.
				\item \textbf{Progress} \\
				Added more AI reviews of tech that I can use:
				Tensor Flow, pytorch, pykov, NLTK, CFG.
				\item \textbf{Summary} \\
				Looked at feedback methods of audio sensors, and AI tools. Language processing may fall out of scope. Its a lot of data.
				Use timelines to implement jokes, break jokes into one sentence in each box, put waits and ambient movement in each joke. Make two timeline boxes to vary the joke\_robot and joke\_human 
			\end{itemize}
		\item{Week 9}
			\begin{itemize}
				\item \textbf{Plans} \\
			testt Say and branching dialogue in choregraphe 
				\item \textbf{Problems} \\
					Organizing large choregraphe functions (timelines are going to be better for things that depend on time, obviously you dunce) 			
				\item \textbf{Progress} \\
			Got say boxes and branching to work, but in a messy way	
				\item \textbf{Summary} \\
				Movements and jokes are going to be best implemented in choregraphe. Algorithm may work by just taking text input, returning text input and the branches just follow the path
			\end{itemize}
		\item{Week 10}
				\begin{itemize}
				\item \textbf{Plans} \\
				Finish progress report
				\item \textbf{Problems} \\
				We should be worried about the progress report, just because it is a lot of work due in a short time.
				\item \textbf{Progress} \\
				Finished Progress report, wrote some comedic devices for comedy. Like, observational humor, Jobs jokes, like uber driver, dance lessons, unemployment
				\item \textbf{Summary} \\
				Further looked at types of jokes, finished Progress report
			\end{itemize}
	\end{itemize}
	\subsubsection{Winter Term}
	\begin{itemize}
		\item{Week 1}
		\item{Week 2}
		\item{Week 3}
		\item{Week 4}
		\item{Week 5}
		\item{Week 6}
		\item{Week 7}
		\item{Week 8}
		\item{Week 9}
		\item{Week 10}
	\end{itemize}

	\subsubsection{Spring Term}
	\begin{itemize}
		\item{Week 1}
		\item{Week 2}
		\item{Week 3}
		\item{Week 4}
		\item{Week 5}
		\item{Week 6}
		\item{Week 7}
		\item{Week 8}
		\item{Week 9}
		\item{Week 10}
	\end{itemize}

	\pagebreak

	\subsection{Anish Asrani}
	\subsubsection{Fall Term}
	\begin{itemize}
		\item{Week 1-2}
			\begin{itemize}
				\item \textbf{Plans} \\
				Get in touch with client and figure out a good time to meet them.
				\item \textbf{Problems} \\
				Finding a suitable time to work for everyone including the client (minor problem).
				\item \textbf{Progress} \\
				Got in touch with teammates. We set up a Slack channel to communicate going forward. 
				\item \textbf{Summary} \\
				Researched many of the projects that I was interested in. Met with Dr. Heather Knight and discussed the scope of her project and got a better idea about it. It definitely seems like an exciting one. Briefly discussed Dr. Mike Bailey's project with him. After lots of consideration and thinking, decided the top 5 projects. Now all that is left is stay put and be anxious until the projects are assigned. 
				We ended up finding a weekly meeting time with the client. Started working on the problem statement.  We will definitely get a much better idea after we meet our client. 
			\end{itemize}
		\item{Week 3}	
			\begin{itemize}
				\item \textbf{Plans} \\
				Work on finishing problem statement.
				\item \textbf{Problems} \\
				Nothing crazy. Just putting all the info we have in words.
				\item \textbf{Progress} \\
				Working on problem statement with my group.
				\item \textbf{Summary} \\
				Met with Heather Knight. Got a lot of information to work with.
			\end{itemize}
		\item{Week 4}
			\begin{itemize}
				\item \textbf{Plans} \\
				TA Meeting and client meeting. Get familiar with Choregraphe. Work on Problem Statement. Research HRI and humor.
				\item \textbf{Problems} \\
				Finding specific papers on robot performances. See what works in humor/what doesn’t. Aspects that make robots more "human"

				\item \textbf{Progress} \\
				Got a bunch of resources from Heather. Body language improves performance. Improved the problem statement significantly
				\item \textbf{Summary} \\
				Interesting meeting and got a lot to work with. We managed to get a fair amount of research about robots and performances. We need to find an appropriate way to link them both. 
				We finished a draft that Heather did not approve. We made a good amount of changes and supported our claims. This should be improved this week.
			\end{itemize}
		\item{Week 5}
			\begin{itemize}
				\item \textbf{Plans} \\
				Write jokes. Get familiar with Choregraphe.
				\item \textbf{Problems} \\
				Writing scripts can be hard. Defining some aspects of our requirements
				\item \textbf{Progress} \\
				Got some ideas for jokes. Made some progress on requirements
				\item \textbf{Summary} \\
				Went over humor at the meeting and saw potential scripts for the robot that we wrote. Worked on the requirements draft. Discussed our problem statement paper and shared what we are doing with Kirsten. Need to continue working on the requirements doc and make a simple choregraphe program over the weekend. 
			\end{itemize}
		\item{Week 6}
			\begin{itemize}
				\item \textbf{Plans} \\
				Work on requirements. Learn choregraphe
				\item \textbf{Problems} \\
				Finding balance in the requirements doc. Research requirements are hard to predict.
				\item \textbf{Progress} \\
				Made a basic set on Choregraphe.
				Wrote significant parts of requirements.
				\item \textbf{Summary} \\
				The week was alright. Some confusion about the requirements due to the research we are trying to do. Overall, there is more clarity than before which is only going to improve over the weeks. 
We managed to write some content for the requirements subject to Heather's approval. 
			\end{itemize}
		\item{Week 7}
			\begin{itemize}
				\item \textbf{Plans} \\
				Analyze technology and literature reviews.
				\item \textbf{Problems} \\
				Figuring out specific technologies we are using, clarifying requirements further.
				\item \textbf{Progress} \\
				Got some good research to help us get started into the depth of things.
				\item \textbf{Summary} \\
				Met the robot. It was great. Discussed our individual research questions. Got a rundown of previous code.
			\end{itemize}
		\item{Week 8}
			\begin{itemize}
				\item \textbf{Plans} \\
				Learn about sensors in the NAO. Research for lit review.
				\item \textbf{Problems} \\
				Hard to test sensors without access to robot. Limited time with bot.
				\item \textbf{Progress} \\
				Found interesting research about HRI.
				\item \textbf{Summary} \\
				Nothing crazy. Work on paper. Learn more software.
			\end{itemize}
		\item{Week 9}
			\begin{itemize}
				\item \textbf{Plans} \\
				Learn more choregraphe. Make scripts. Account for pauses during performance
				\item \textbf{Problems} \\
				Animating the robot can take a lot of time.
				\item \textbf{Progress} \\
				Getting the hang of choregraphe.
				\item \textbf{Summary} \\
				Thanksgiving dinner was great.
			\end{itemize}
		\item{Week 10}
			\begin{itemize}
				\item \textbf{Plans} \\
				Work on progress report and script.
				\item \textbf{Problems} \\
				Design document and progress report within a few days
				\item \textbf{Progress} \\
				Record footage, wrote scripts
				\item \textbf{Summary} \\
				Recordings, scripts, meetings, papers = dead week
			\end{itemize}
	\end{itemize}
	\subsubsection{Winter Term}
	\begin{itemize}
		\item{Week 1}
			\begin{itemize}
				\item \textbf{Plans} \\
				Get back in the flow of things for winter. Play around with robot.
				\item \textbf{Problems} \\
				Not a whole lot this week.
				\item \textbf{Progress} \\
				Made a few animations and script
				\item \textbf{Summary} \\
				Not a whole lot but warm up to the robot after it was away for all of winter break
			\end{itemize}
		\item{Week 2}
			\begin{itemize}
				\item \textbf{Plans} \\
				Write/execute jokes. Improve design doc
				\item \textbf{Problems} \\
				Writing jokes can be hard
				\item \textbf{Progress} \\
				Got a few joke ideas.
				\item \textbf{Summary} \\
				Churn out jokes - keep working on experiment (IRB) draft)
			\end{itemize}
		\item{Week 3}
			\begin{itemize}
				\item \textbf{Plans} \\
				Similar to last week. Write jokes. Improve doc.
				\item \textbf{Problems} \\
				Joke writing.
				\item \textbf{Progress} \\
				Slow but steady progress.
				\item \textbf{Summary} \\
				More paperwork for design document.
			\end{itemize}
		\item{Week 4}
			\begin{itemize}
				\item \textbf{Plans} \\
				Explore sensors further.
				\item \textbf{Problems} \\
				Hard to test and predict how robot will behave in different environments.
				\item \textbf{Progress} \\
				Got a sound tracking system working that still needs more testing
				\item \textbf{Summary} \\
				Executed an sound following module on the robot. The robot now turns its head toward wherever a sound is coming from. Useful for crowd work.
			\end{itemize}
		\item{Week 5}
			\begin{itemize}
				\item \textbf{Plans} \\
				Poster, progress vid, try using NAOqi SDK.
				\item \textbf{Problems} \\
				NAOqi SDK is hard to work with.
				\item \textbf{Progress} \\
				Worked on some scripts for facial recognition.
				\item \textbf{Summary} \\
				Collaborated on some robot scripts - worked on poster draft.
			\end{itemize}
		\item{Week 6}
			\begin{itemize}
				\item \textbf{Plans} \\
				Focus on midterm report/video.
				\item \textbf{Problems} \\
				Recording and getting robot footage takes time.
				\item \textbf{Progress} \\
				Video went alright.
				\item \textbf{Summary} \\
				Did midterm report. Record/edit videos - churn out content
			\end{itemize}
		\item{Week 7}
			\begin{itemize}
				\item \textbf{Plans} \\
				Understand sensors and API usage.
				\item \textbf{Problems} \\
				Integrating sensors with other animations can break the animations.
				\item \textbf{Progress} \\
				Head/sound tracking works if there are no keyframes stored in head
				\item \textbf{Summary} \\
				Working further on using crowd work components like comparing volume levels from the API.
			\end{itemize}
		\item{Week 8}
			\begin{itemize}
				\item \textbf{Plans} \\
				Communicate with the robot from local computer.
				\item \textbf{Problems} \\
				Compatibility issues with OSX and Python versions.
				\item \textbf{Progress} \\
				Trying different Python installations. It was hard to work with.
				\item \textbf{Summary} \\
				Learned about QiMessaging - communicating with the robot and parsing values
			\end{itemize}
		\item{Week 9}
			\begin{itemize}
				\item \textbf{Plans} \\
				Audio level sensing from the API.
				\item \textbf{Problems} \\
				Testing without robot is hard since it is away this week.
				\item \textbf{Progress} \\
				Have an idea of how the API works - need to test.
				\item \textbf{Summary} \\
				Worked on some aspects of communicating data via the robot by reading the audience. Still needs further testing.
			\end{itemize}
		\item{Week 10}
			\begin{itemize}
				\item \textbf{Plans} \\
				Progress video/paper - final audio sensing numbers.
				\item \textbf{Problems} \\
				NAOqi documentation is outdated - Some documentation randomly jumps from Python to C++.
				\item \textbf{Progress} \\
				Had our first demo with people - got a script to sense volume levels.
				\item \textbf{Summary} \\
				Need to integrate the audio level script into Choregraphe. Had a demo, couple bugs showed up but nothing crazy. Need more human testing
			\end{itemize}
	\end{itemize}

	\subsubsection{Spring Term}
	\begin{itemize}
		\item{Week 1}
			\begin{itemize}
				\item \textbf{Plans} \\
				Implement sound levels on robot/send message to robot.
				\item \textbf{Problems} \\
				No robot. Power brick is broken.
				\item \textbf{Progress} \\
				New brick ordered
				\item \textbf{Summary} \\
				Discussed the remaining aspects of our project on a high level - have a fair idea of how this quarter is going to go for - makers faire \& expo.
			\end{itemize}
		\item{Week 2}
			\begin{itemize}
				\item \textbf{Plans} \\
				Figure out buttons on RasPi. HWeekend.
				\item \textbf{Problems} \\
				No robot power still - brick dead
				\item \textbf{Progress} \\
				Got a more fleshed out plan from Heather (for the term).
				\item \textbf{Summary} \\
				Discussed a lot of specifics with Heather individually, got more major goals lined up.
			\end{itemize}
		\item{Week 3}
			\begin{itemize}
				\item \textbf{Plans} \\
				Added a joke and worked further toward robot communication
				\item \textbf{Problems} \\
				NAO OS is hard to play with.
				\item \textbf{Progress} \\
				Close to meeting research requirements + Maker Faire set
				\item \textbf{Summary} \\
				Pi controller specs - joke model specs
			\end{itemize}
		\item{Week 4}
			\begin{itemize}
				\item \textbf{Plans} \\
				TCP communication between Pi and Robot.
				\item \textbf{Problems} \\
				NAO operating system feels very restricted. Hard to mess with.
				\item \textbf{Progress} \\
				Got robot to launch behaviors from the controller.
				\item \textbf{Summary} \\
				Solid progress on the controller communication. Preparing for Maker Fair! 
			\end{itemize}
		\item{Week 5}
			\begin{itemize}
				\item \textbf{Plans} \\
				Get some good feedback on Maker Fair
				\item \textbf{Problems} \\
				Maker Fair was loud. People struggled to listen to the robot. Noticed a major flaw in the system.
				\item \textbf{Progress} \\
				Optimized some scripts for Maker Fair. Good indicator for expo.
				\item \textbf{Summary} \\
				Maker Fair went well despite some drawbacks with listening to sound. We got some really good feedback. Had to improvise and have a backup text for people to read while the robot performed.
			\end{itemize}
		\item{Week 6}
			\begin{itemize}
				\item \textbf{Plans} \\
				Back to the drawing board - change a couple audience interactions to meet expo environment.
				\item \textbf{Problems} \\
				Hard to predict expo environment and what can go wrong.
				\item \textbf{Progress} \\
				Discussed issues with Heather. She had a few solutions for us to try at Expo.
				\item \textbf{Summary} \\
				Always plan for failure. If no feedback - move ON, don't get stuck.
			\end{itemize}
		\item{Week 7}
			\begin{itemize}
				\item \textbf{Plans} \\
				Optimize scripts for expo. Think of jokes.
				\item \textbf{Problems} \\
				Almost a year later - jokes are still hard
				\item \textbf{Progress} \\
				Got more ideas from Heather. 
				\item \textbf{Summary} \\
				Coming real close to expo. Tweaking scripts to match expo presentations.
			\end{itemize}
		\item{Week 8}
			\begin{itemize}
				\item \textbf{Plans} \\
				Optimize optimize optimize.
				\item \textbf{Problems} \\
				Expo jitters - hard to predict what's going to happen. Will people hear the robot? Will jokes go across well?
				\item \textbf{Progress} \\
				Wrote a python script for the entire expo performance to tie everything together and branch off based on audience feedback.
				\item \textbf{Summary} \\
				EXPO - it went way better than expected. People enjoyed our project - most of the jokes worked well. Interactions worked well. Nothing broke during expo.
			\end{itemize}
		\item{Week 9}
			\begin{itemize}
				\item \textbf{Plans} \\
				Got research tasks to do from Heather. Figure out specifics for that. Start planning final paper/presentation
				\item \textbf{Problems} \\
				Nothing yet. Post expo.
				\item \textbf{Progress} \\
				Plotted out what is left to do for capstone over next two weeks.
				\item \textbf{Summary} \\
				Expo is over. Research assignments from Heather. Heard about the final documentation needed for the class.
			\end{itemize}
		\item{Week 10}
			\begin{itemize}
				\item \textbf{Plans} \\
				Finalize statistics for research, finish doc, video.
				\item \textbf{Problems} \\
				Crunch time.
				\item \textbf{Progress} \\
				Video done, more work needed on paper.
				\item \textbf{Summary} \\
				Final week - wrap up loose ends. Nothing too crazy.
			\end{itemize}
	\end{itemize}

	\pagebreak


	\subsection{Arthur Shing}
	\subsubsection{Fall Term}
	\begin{itemize}
		\item{Week 1-2}
			\begin{itemize}
				\item \textbf{Plans} \\
				Lorem
				\item \textbf{Problems} \\
				Ipsum
				\item \textbf{Progress} \\
				Merol
				\item \textbf{Summary} \\
				Muspi
			\end{itemize}
		\item{Week 3}
		\item{Week 4}
		\item{Week 5}
		\item{Week 6}
		\item{Week 7}
		\item{Week 8}
		\item{Week 9}
		\item{Week 10}
	\end{itemize}
	\subsubsection{Winter Term}
	\begin{itemize}
		\item{Week 1}
		\item{Week 2}
		\item{Week 3}
		\item{Week 4}
		\item{Week 5}
		\item{Week 6}
		\item{Week 7}
		\item{Week 8}
		\item{Week 9}
		\item{Week 10}
	\end{itemize}

	\subsubsection{Spring Term}
	\begin{itemize}
		\item{Week 1}
		\item{Week 2}
		\item{Week 3}
		\item{Week 4}
		\item{Week 5}
		\item{Week 6}
		\item{Week 7}
		\item{Week 8}
		\item{Week 9}
		\item{Week 10}
	\end{itemize}

	\pagebreak
