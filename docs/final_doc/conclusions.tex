
\section{Final Team Conclusions}
\subsection{Kevin Talik}

\begin{itemize}
\item{\textbf{What technical information did you learn?}}

    I learned how to perform research, and define gaps in current research.
    Research is not entirely about doing something completely new, but searching through current research to find a "Gap".
    There can be a lot of pressure to deliver results from research, but it is far more important to understand the premise of collecting data.
    These projects can scale out of scope if the final method for collecting information is not clearly defined in the beginning.

    Additionally, I learned that there are tasks robots can do, and tasks robots should do.
    There is never a situation where you should value a robot over a human.
    Comparing "Robots vs Humans" insinuates that there is a moment where they could be equal.
    AI is not a species, but a tool that humans can use as an end through means.

    Robot comedy is identical to ventriloquism.
    We write the words and motions that the robot re-enacts.
    Kids, and people who do not understand \texttt{if-else} statements think that AI is making decisions.
    If the robot offends people with a joke, we, as the comedians, must take ownership of the robot's decisions.
    A robot should not be making ethical decisions of what to say in front of a crowd.


\item{\textbf{What non-technical information did you learn?}}


    People laugh when they see something, and are surprised and \textit{comfortable}.
    Also, people force a laugh when they are surprised and \textit{uncomfortable}.
    Listening to laughter alone can be a bad metric for comedy, as it is hard to tell when people are laughing at you, or with you.

    There is no truth to comedy, it is a written art that people study to find what makes people comfortable.
    When people are laughing, they are not paying attention to the computer that is inside of the robot.
    Comedy is a \textit{very} persuasive form of communication, and finding when a human is comfortable is not a task a robot should have.


\item{\textbf{What have you learned about project work?}}

    Project work is primarily time-dependent; however, there are always different expectations about how much time a person can give.
    Software must be planned in a scalable manner so that it does not fall ahead or behind scope during implementation.
    The work a person does must be human-readable, and easy to understand.
    This is so that when another person looks at what you've done, it is easier and faster for them to connect larger concepts.
    This must be balanced with a healthy, and sustainable personal life balance.
    Stress is a normal thing that humans have to deal with, and avoiding it brings more stress.




\item{\textbf{What have you learned about project management?}}

    The fundamentals of a good team is honest, and healthy communication.
    Our team functioned well when we all treated each other with respect.
    Patience is important in software. It is unreasonable to assume everyone knows everything.
    Lift as you climb, and help your team.


\item{\textbf{What have you learned about working in teams?}}

    For the success of a comedy show, a team must be heavily involved with every portion of a performance.
    Machines that are built alone, work alone. Coding standards exist so that similar looking projects look the same.
    "camelCasing" or "underscore\_casing" is unified so that a team can write similar looking code.
    There is always enough time for one person to do an entire project, but it is not a single-human task.
    Our final project got much better once we worked together in the same room.


\item{\textbf{If you could do it all over, what would you do differently?}}

    I think that our performance needed to have equally distributed testing (performing jokes) with implementing (writing jokes).
	The only way to test a joke is to tell it to a realistic audience.
	Often, what we thought was funny in a joke was different from what the audience thought.
	 The only way to test a joke is to tell it to a realistic audience.
	 Often, what we thought was funny in a joke was different from what the audience thought.
	 Also, I probably should not have started smoking cigarettes, and started to take care of my health.

	 \end{itemize}


\pagebreak

\subsection{Anish Asrani}

\begin{itemize}
\item{\textbf{What technical information did you learn?}}

    I learned a lot about intricacies and quirks of doing research. There is so much that goes into performing a research -
    how it is defined, how various areas are explored, how research decisons are made. It was almost overwhelming when I first
    started working on it, but I came around to it. It gave me a different perspective on how I look at research now.
    Even what may seem like a "small" research has so much going on behind the curtains.

\item{\textbf{What non-technical information did you learn?}}

    Human-robot interaction is a field that has so many different outlooks. One of those is the psychology behind it.
    I learned a lot about a field that I did not know existed. Other than that, I learned about how a performance works,
    and how jokes are structures (jokes are hard).

    It drove me to give improv comedy a shot. It is something I have enjoyed doing for the past year, and I plan to continue
    doing it in the upcoming years.

    This would have been the first time that I worked in a team for longer than a few weeks. So it taught me how to better work
    in teams as well.

\item{\textbf{What have you learned about project work?}}

    Prioritizing goals is very important. Figure out what is most important at a given point, and then focus on that alone.
    It also helps to have everyone work on the same page, rather than digging away in different directions. If everyone is working on the same page,
    they can build a coherent system even if there are disagreements to begin with.

    Constant communication is also vital in a project. If everyone is aware what everyone else is doing at a given time, it helps avoid
    duplicate and/or incoherent work.

\item{\textbf{What have you learned about project management?}}

    As mentioned earlier, managing and prioritizing goals really helps put together a more consistent project. It is important to define
    these priorities early in the process to ensure everyone is working toward a common goal.

    Everyone on a team will have different strengths and weaknesses, different times when they are more productive than others.
    Considering those factors is important to get the best out of everyone and as a result, the project as a whole.

    Considering all of this, setting timelines and estimating when certain goals will be met is difficult especially if the technology
    used is something that you have never encountered before. This is something I learned over the course of the year.
    It will definitely be something I spend more time analyzing when starting a larger-scale project.


\item{\textbf{What have you learned about working in teams?}}

    There are so many different perspectives that will come up when working with teams. It is important for everyone to be
    onboard for a certain task to be successfully accomplished. If people are swaying off in different directions,
    and never come to an agreement, it is hard for the project to be successful.

    I learned more about those different perspectives and having an open mind about doing things the way I usually wouldn't.


\item{\textbf{If you could do it all over, what would you do differently?}}

    I could have thought of ways to put our robot sets in front of a real human audience as often as possible, even if we weren't confident
    of it doing well. It was only after we got some real feedback from an audience at Maker Faire/Expo, we started to understand some of the flaws
    in our system. If we had started demo-ing the smallest of sets to a small audience, we could have gotten more feedback in order to
    iterate toward a better performing and more coherent system.

    I would have also tried to know more about the research aspect of the project \textit{right when I started working on it.} More specifically,
    what kind of data is useful in research, and what is not.

\end{itemize}
\clearpage

\subsection{Arthur Shing}
\begin{itemize}
\item{\textbf{What technical information did you learn?}}
  In terms of technical information, I learned about researching and using APIs that have little documentation, and implementing them into a project. I also learned about the research process, and how project goals, requirements, or expectations are adjusted throughout the process.
  Additionally, I learned about text-to-speech software limitations, and how some of these limitations can be overridden through manipulations that are provided by the software.
  I also learned about the limitations of audio sensors, such as their inability to detect differences due to the acoustics of their environment. I also learned how to use Choregraphe, and how to program a NAO robot through it.
\item{\textbf{What non-technical information did you learn?}}
  Non-technical information I learned includes our expo audience analysis results. We found that people who have never seen the robot before enjoyed the robot's interactions, physical appearance, and mannerisms. We also found that people found audience interaction to be the most entertaining part of having a robot comedian, and that the context of a joke (robot/human, jobs/aging/romance) have little to do with the reception of the joke.

  In terms of comedy, I learned about routines that stand-up comedians use to up their entertainment value. I also learned about how the structure of a comedy set can affect its reception.

  In terms of Human-Robot Interaction, I learned about how quirks and mannerisms can seemingly give like to a robot's character, and make interactions more enjoyable.
\item{\textbf{What have you learned about project work?}}
  In terms of working in projects, I learned that it is helpful to have everyone on the team on the same page at all times. I also learned about the necessity of having group members help each other outin times of need, as the project as a whole often depends on all the separate parts working together. I learned that it is also important to do your best to stick with the projected timeline and deadlines, but often project progress may fall short and goal expectations are set too high. I learned about the importance of acknowledging your own limits, with regards to time, productivity, and skill in working on large projects. I also learned about the importance of reviewing the client's requirements, to have well-attuned priorities.
\item{\textbf{What have you learned about project management?}}
  In terms of project management, I learned that prioritizing goals, and splitting up work can make projects more productive. I learned that underestimating our team's capabilities could result in more productive uses of time, more room for ideas to foster, and a higher quality product. On the other hand, overestimating our capabilities led to our resources being stretched too thin across a multitude of aspects of the project, and made the process rushed, stressful, and not of high quality.

\item{\textbf{What have you learned about working in teams?}}
  I have learned that it is important to listen to what your teammates are saying, and to be on the same page as everyone else on the team. I also learned that it is important to rely on aid from your teammates when there are aspects of the project that you are having difficulty with.

\item{\textbf{If you could do it all over, what would you do differently?}}
  I would have chosen to negotiate with our client so that only one or two research areas would be our goal, as three was too much for our team. I also would have made a larger effort to grab joke ideas from friends, and would have chosen to change my research area to robot vs humanlike movements. It was much too hard of a literary problem to create variations of jokes that weren't created to have variations. I also would have spent more time developing with my teammates in their portions of the project, as it was sometimes difficult to throw our separate portions of the project together.


\end{itemize}
