\documentclass[onecolumn, draftclsnofoot,10pt, compsoc]{IEEEtran}
\usepackage{graphicx}
\usepackage{url}
\usepackage{setspace}
\usepackage{pdflscape}

\usepackage{tikz}
\usetikzlibrary{%
	arrows, backgrounds, calc,%
    patterns, positioning, shapes.geometric%
}
\RequirePackage{pgfcalendar}
\usepackage{pgfgantt}

\usepackage{geometry}
\geometry{textheight=9.5in, textwidth=7in,margin=0.75in}
\bibliographystyle{IEEEtran}

% 1. Fill in these details
\def \CapstoneTeamName{			AKARobotics}
\def \CapstoneTeamNumber{		13}
\def \GroupMemberOne{			Anish Asrani}
\def \GroupMemberTwo{			Kevin Talik}
\def \GroupMemberThree{			Arthur Shing}
\def \CapstoneProjectName{		How to Make an Effective Robot Comedian}
\def \CapstoneSponsorCompany{	Oregon State University}
\def \CapstoneSponsorPerson{		Heather Knight}

% 2. Uncomment the appropriate line below so that the document type works
\def \DocType{
% Problem Statement
				% Requirements Document
				Literature Review
				%Design Document
				%Progress Report
				}

\newcommand{\NameSigPair}[1]{\par
\makebox[2.75in][r]{#1} \hfil 	\makebox[3.25in]{\makebox[2.25in]{\hrulefill} \hfill		\makebox[.75in]{\hrulefill}}
\par\vspace{-12pt} \textit{\tiny\noindent
\makebox[2.75in]{} \hfil		\makebox[3.25in]{\makebox[2.25in][r]{Signature} \hfill	\makebox[.75in][r]{Date}}}}
% 3. If the document is not to be signed, uncomment the RENEWcommand below
%\renewcommand{\NameSigPair}[1]{#1}

%%%%%%%%%%%%%%%%%%%%%%%%%%%%%%%%%%%%%%%

\begin{document}

\bstctlcite{IEEEexample:BSTcontrol}

\begin{titlepage}
    \pagenumbering{gobble}
    \begin{singlespace}
        \hfill
        % 4. If you have a logo, use this include graphics command to put it on the coversheet.
        %\includegraphics[height=4cm]{CompanyLogo}
        \par\vspace{.2in}
        \centering
        \scshape{
             \huge CS Capstone \DocType \par
            {\large\today}\par
            \vspace{.5in}
            \textbf{\Huge\CapstoneProjectName}\par
            \vfill
            {\large Prepared for}\par
            \Huge \CapstoneSponsorCompany\par
            \vspace{5pt}
            {\Large\NameSigPair{\CapstoneSponsorPerson}\par}
            {\large Prepared by }\par
            Group\CapstoneTeamNumber\par
            % 5. comment out the line below this one if you do not wish to name your team
            \CapstoneTeamName\par
            \vspace{5pt}
            {\Large
                \NameSigPair{\GroupMemberOne}\par
                \NameSigPair{\GroupMemberTwo}\par
                \NameSigPair{\GroupMemberThree}\par
            }
            \vspace{20pt}
        }
        \begin{abstract}
        % 6. Fill in your abstract




A comedian can observe an audience and improvise a delivery of a joke to connect the audience to the content. This makes the experience more authentic and genuine for the observer. The purpose of this project is to to discover what makes an entertaining interaction by studying a robot that performs comedy. We propose that a performance is enhanced when (1) the comedian interacts spontaneously with the audience, (2) the comedian has and conveys a coherent, well-developed character, and (3) the comedian adapts its act to cater to an audience based on their reaction. This document covers the technical requirements for our project, as well as a description of software, hardware, and outside limitations.


        \end{abstract}
    \end{singlespace}
\end{titlepage}
\newpage
\pagenumbering{arabic}
\tableofcontents
% 7. uncomment this (if applicable). Consider adding a page break.
% \listoffigures
\listoftables
\clearpage

\section{Introduction}
Eight Lessons learned about Non-verbal Interactions through Robot Theater \cite{KnightEightLessons:2011}.  Robot theater is one of the newer areas to research Human-Robot Interaction (HRI). There were various aspects about the study that could be used to improve robot comedy. The lessons learned were: 

\section{Convey Intentionality}
The robot using relatable and appropriate gestures help improve the communication with the audience. These actions are considered relatable by humans since they can predict what the robot is going to do. Another aspect that could the robot be relatable is to display empathy while performing certain actions.
	
\section{No Mind without Body}
	Robots can be more expressive if they use their physicality. 
	
\section{Physicality and Motion}
	The audience should be able to interpret the robot?s non-verbal behaviors. The human brain maps the actions on to itself. This leads to the brain simulating those actions in the best way possible.
	
\section{Outward Emotional Communication Trumps Inward Experience}
	Often, simplicity is the clearest way for a robot to communicate intention. If the humans encourage goodwill toward the robot, it could result in a sense of accomplishment for the humans involved.
	
\section{Gulf between Props and Character}
	Robots should be considered to be agents (entities) and less like a prop just standing up on stage. Until now, robots have been considered as objects. This is due to the robots lacking believable and human-like actions. The various gestures and aspects of interaction is the difference between a social being and an object.
	
\section{Good Actors outweigh Bad Actors}
	Having multiple robots interacting with each other, or having a human-robot duo could be effective in communicating with the audience. In case of the human-robot duo, the human makes up for the robot?s lack of control and unpredictability. 
	
\section{Acknowledgement/Learning}
	Human audiences are cognizant of human social behaviors. The audience can provide real time feedback. This feedback can be used to maximize audience?s enjoyment levels. The robot can constantly read these enjoyment levels and update the attributes of audience likes and dislikes. A joke should be given enough of a pause for the audience to take it in. Starting the next joke early breaks the flow and does not give the audience a chance to appreciate the joke. The pause could be filled with the robot gazing at the audience and posing. This will help develop a good rhythm for each individual joke.
	
\section{Humor makes people like the robot}
	Humor is a common ground that knows no bounds. When a robot performs comedy, it establishes that common ground with the audience. Once this common ground is established, the audience would be willing to forgive the robot?s shortcomings and failures. If the robot is able to recognize its failures, it would add to the audience liking the robot. 
	
\section{Conclusion}
	In progress..
	
	
	

\pagebreak


\bibliographystyle{IEEEtran}
\bibliography{refs}
\end{document}
