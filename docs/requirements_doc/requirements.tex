\documentclass[onecolumn, draftclsnofoot,10pt, compsoc]{IEEEtran}
\usepackage{graphicx}
\usepackage{url}
\usepackage{setspace}
\usepackage{pdflscape}

\usepackage{tikz}
\usetikzlibrary{%
	arrows, backgrounds, calc,%
    patterns, positioning, shapes.geometric%
}
\RequirePackage{pgfcalendar}
\usepackage{pgfgantt}

\usepackage{geometry}
\geometry{textheight=9.5in, textwidth=7in,margin=0.75in}
\bibliographystyle{IEEEtran}

% 1. Fill in these details
\def \CapstoneTeamName{			Laughtimus Prime}
\def \CapstoneTeamNumber{		13}
\def \GroupMemberOne{			Anish Asrani}
\def \GroupMemberTwo{			Kevin Talik}
\def \GroupMemberThree{			Arthur Shing}
\def \CapstoneProjectName{		How to Make an Effective Robot Comedian}
\def \CapstoneSponsorCompany{	Oregon State University}
\def \CapstoneSponsorPerson{		Heather Knight}

% 2. Uncomment the appropriate line below so that the document type works
\def \DocType{
% Problem Statement
				Requirements Document
				%Technology Review
				%Design Document
				%Progress Report
				}

\newcommand{\NameSigPair}[1]{\par
\makebox[2.75in][r]{#1} \hfil 	\makebox[3.25in]{\makebox[2.25in]{\hrulefill} \hfill		\makebox[.75in]{\hrulefill}}
\par\vspace{-12pt} \textit{\tiny\noindent
\makebox[2.75in]{} \hfil		\makebox[3.25in]{\makebox[2.25in][r]{Signature} \hfill	\makebox[.75in][r]{Date}}}}
% 3. If the document is not to be signed, uncomment the RENEWcommand below
%\renewcommand{\NameSigPair}[1]{#1}

%%%%%%%%%%%%%%%%%%%%%%%%%%%%%%%%%%%%%%%

\begin{document}

\bstctlcite{IEEEexample:BSTcontrol}

\begin{titlepage}
    \pagenumbering{gobble}
    \begin{singlespace}
        \hfill
        % 4. If you have a logo, use this include graphics command to put it on the coversheet.
        %\includegraphics[height=4cm]{CompanyLogo}
        \par\vspace{.2in}
        \centering
        \scshape{
             \huge CS Capstone \DocType \par
            {\large\today}\par
            \vspace{.5in}
            \textbf{\Huge\CapstoneProjectName}\par
            \vfill
            {\large Prepared for}\par
            \Huge \CapstoneSponsorCompany\par
            \vspace{5pt}
            {\Large\NameSigPair{\CapstoneSponsorPerson}\par}
            {\large Prepared by }\par
            Group\CapstoneTeamNumber\par
            % 5. comment out the line below this one if you do not wish to name your team
            \CapstoneTeamName\par
            \vspace{5pt}
            {\Large
                \NameSigPair{\GroupMemberOne}\par
                \NameSigPair{\GroupMemberTwo}\par
                \NameSigPair{\GroupMemberThree}\par
            }
            \vspace{20pt}
        }
        \begin{abstract}
        % 6. Fill in your abstract




A comedian can observe an audience and improvise a delivery of a joke to connect the audience to the content. This makes the experience more authentic and genuine for the observer. The purpose of this project is to to discover what makes an entertaining interaction by studying a robot that performs comedy. We propose that a performance is enhanced when (1) the comedian interacts spontaneously with the audience, (2) the comedian has and conveys a coherent, well-developed character, and (3) the comedian adapts its act to cater to an audience based on their reaction. This document covers the technical requirements for our project, as well as a description of software, hardware, and outside limitations.


        \end{abstract}
    \end{singlespace}
\end{titlepage}
\newpage
\pagenumbering{arabic}
\tableofcontents
% 7. uncomment this (if applicable). Consider adding a page break.
% \listoffigures
\listoftables
\clearpage

\section{Background}
Robot interactions can learn a lot from stand-up comedy. A stand-up performance has a basis of scripted content, from which the comedian delivers jokes to engage the audience. A good comedian can read the audience, and adapt delivery based on the mood of the room \cite{talkingFunny}. In social robotics, when a robot shares a space with a human, an interaction can influence the humans opinion of the robot. Additionally, evident character traits presented (through dialogue and non-verbal motion) by the machine can anthropomorphize itself, making it easier and more enjoyable to connect with for the human \cite{KnightEightLessons:2011}.

In previous studies, robots utilizing non-verbal communication, as well as attempts at adaptive robots comedians have been done. In particular, Heather Knight has observed the importance of character and spontaneous interactions in creating effective comedy \cite{KnightEightLessons:2011}. However, there is little research on the actual effectiveness of character and spontaneous interactions \cite{KatevasRobot:2014}. This project will aim to examine the effectiveness of character and spontaneous interactions in robot comedy.


\section{Hypotheses}

Our research will be guided by the following questions:
\begin{enumerate}[\IEEEsetlabelwidth{6)}]
\item How can the robot integrate the audience to make them feel like a part of the performance?
\item How can the robot convey and a coherent and well-developed character?
\item How can the robot adapt and influence to the audience?
\end{enumerate}

We hypothesize that a clearly evident character for the robot will positively influence the performance, and will enage the audience better than a set with no characterizations. In a previous study of robot comedy \cite{RobotComedyLab:2015}, Katevas found that when a robot engaged the audience through eye contact, the audience was more receptive to the performance. Eye contact from the robot is important, as it is a non-verbal cue for direct interaction. The audience members can identify that the machine is making an attempt to engage with specific members of the audience. This correlates with Dr. Heather Knight's \cite{KnightEightLessons:2011} research that outward communication results in a feeling of accomplishment for a human observer.

Dr. Heather Knight has researched Robot Theatre as a metaphor for HRI, and found that when a robot is viewed as an agent (characterized object), the interaction arc with a human is stronger than if the robot is treated as a prop. When a robot conveys an intelligence and characterizations of itself, the audience can connect to the robot as an agent, and not inanimate.

Expressing character will be acheived from the robot under the "theory of the mind" \cite{leslie}. Understanding the agents metarepresentation of a behavior is helpful for the audience in relating to the robot's intent, desires and knowledge. A robot attempts this to understand and relate its desires and intent to better relate to the audience \cite{theoryOfMindRobots}.

\section{Research Approach}
This project will be carried out in three phases; A learning and exploration phase in Fall term, a prototyping and testing phase in the Winter, and our evaluation phase Spring term.

\subsection{Learning/Exploration Phase}
This phase of our development will focus on understanding Social Robotics and the technology of the robot. The three of us will become familiar with stand-up comedy and the dynamics of an audience-comedian interaction. The NAO robot behaviors are programmed in the software Choreographe, which has an API for python. We will test primitive scripts of decision making and non-verbal behavior. This is to learn how the coding environment works and to familiarize ourselves with hardware limitations. Additionally, we will learn to work with the sensors on the robot, and how they function. The sensors in the NAO will be how the virtual audience model is generated. To become familiar with the format of a stand-up performance, we intend to study jokes and comedy devices. A large gap in current Robot Comedy is adaptive audience interaction and witty, spontaneous jokes \cite{KatevasRobot:2014}. The difficulty in this process is effectively understanding an audience model, and timing a coherent joke that accurately relates to the audience.

\subsection{Prototyping/Testing Phase}
In the prototyping and testing phase, we will develop early prototypes based on the preliminary research questions created in the previous phase. As of the time of writing, these questions include the effectiveness of implementing crowdwork and the effectiveness of implementing a discernable character. Crowd-work will involve simple audience sensing, as well as jokes that incorporate a measurement of response from the audience. Character implementation will involve testing the differences in effectiveness of robot vs human joke delivery, and the effectiveness of robo-centric jokes.

As a stretch goal, we also hope to prototype and test the effectiveness of adapting a set to the audience, using intelligent calibration of the sensors. These prototypes will be in the form of set scripts, and will be tested in front of a small sample of humans, or in the form of a video recording. Feedback from our testing will influence the direction of our prototyping, meaning that the implementation of our research questions may vary according to the response.

\subsection{Evaluation Phase}
While doing the research, we will perform 9 shows with audiences ranging from 30-50 people. Each stand-up performance, or set, will contain bits, or sections of content that will be categorized as crowd work, characterization dialogue, and jokes. We will be testing on a live human audience to learn the effectiveness of each bit in a set. Based of the effectiveness of each set, we will modify the set and behavior of the robot.

\section{Methods}
We will make a virtual audience model from the senors on the NAO. Using this model, we can identify the mood of the audience to determine what each response is to each segment of the stand-up set. This will help determine which bits of the set were effective. Multiple robot personalities will be tested to see what kind of character appeals to the audience. The audience will also be surveyed to determine what aspects of the performance were enjoyable and what was not.


Other studies by Katevas et al. \cite{KatevasRobot:2014} that involved evaluating the social dynamics of a live performance by a robot have used SHORE\textsuperscript{TM} vision framework software to analyze and detect faces in the audience. SHORE\textsuperscript{TM} allows for facial expression recognition, estimated age, gender, and eye or mouth openings \cite{SHORE}, giving the study a heterogenous audience model. These allowed for the robot to interact directly with specific audience members. However, usage of SHORE\textsuperscript{TM} involves expenses and funds that are unavailable to us, so we will encounter behavioral limitations dealing with a homogenous audience model.

The effectiveness of our robot comedian will be evaluated by human enjoyment levels. Specifics regarding measurements and analysis will be later discussed with the client. Some possible methods include handing out surveys for the audience to fill out, which may include questions regarding the subjective reception of the robot comedian. Additionally, behavioral statistics may be used to evaluate the effectiveness of the comedy.

\section{Potential Discussions}


Relatable and appropriate gestures significantly helps improve communication between the robot and the audience. If the actions are predictable, humans can relate to the robot {\cite{KnightEightLessons:2011}}.
When watching someone perform an action, the human brain maps the actions onto itself and simulates the action in the best way possible. This is a physiological experience that should be replicated by the robot in order to enhance relatability. Simplicity is important as well.

The robot being portrayed as a living character rather than just an object that is kept up on stage improved the experience. Having believable interactions can enhance the feeling of a living character.
The goal of the audience tracking using sensors is to maximize enjoyment. The enjoyment levels were be read by the robot and used to modify upcoming jokes.

Pausing and letting the audience laugh is vital as well. Starting the next joke too early can break the rhythm and leave the audience baffled. Looking around and body poses should be used to fill the pause {\cite{KnightEightLessons:2011}}.

% \subsection{Timeline}
\pagebreak
\begin{landscape}

\begin{table}
	\begin{ganttchart}[
		hgrid,
		vgrid=true]{1}{27}

		\gantttitle{Title}{27} \\
		\gantttitle{Fall}{10}
		\gantttitle{Winter}{10}
		\gantttitle{Spring}{7} \\
		\gantttitlelist{1,...,10,1,2,3,4,5,6,7,8,9,10,1,2,3,4,5,6,7}{1} \\
		\ganttgroup{Learning \& Exploration}{1}{10} \\
		\ganttbar{Learn Choregraphe}{3}{8} \\
		\ganttlinkedbar{Supplementary Scripts}{8}{10} \\
		\ganttbar[name=Research]{Research Comedy \& HRI}{1}{7} \\
		\ganttmilestone[name=M1]{Evaluate Research Questions}{7} \\
		\ganttgroup{Prototyping \& Testing}{11}{20} \\
		\ganttbar[name=S2]{Developed Scripts}{11}{13} \\
		\ganttlinkedbar[name=T1]{Test w/ Friends}{13}{16} \\
		\ganttlinkedbar[name=T2]{Video Tests}{16}{20} \\
		\ganttbar[name=Q]{Tweak R1, R2, R3 implementations}{13}{20} \\
		\ganttmilestone[name=M2]{Robot Stand-up Set Completed}{20} \\
		\ganttgroup{Live Testing \& Evaluation}{21}{27} \\
		\ganttbar[name=T3]{Live Testing}{21}{27} \\
		\ganttbar{Evaluate Performance}{21}{27}

		\ganttlink{Research}{M1}
		\ganttlink{M1}{S2}
		\ganttlink{Q}{M2}
		\ganttlink{M2}{T3}



		% \ganttlink{elem2}{elem3}
		% \ganttlink{elem3}{elem4}
	\end{ganttchart}
	\caption{A gantt chart showing the projected timeline of the project.}
	\label{Gantt Chart}

\end{table}

\end{landscape}
\pagebreak


\bibliographystyle{IEEEtran}
\bibliography{refs}
\end{document}
